% !TEX encoding = UTF-8
% !TEX TS-program = pdflatex
% !TEX root = ../tesi.tex

%**************************************************************
% Sommario
%**************************************************************
\cleardoublepage
\phantomsection
\pdfbookmark{Sommario}{Sommario}
\begingroup
\let\clearpage\relax
\let\cleardoublepage\relax
\let\cleardoublepage\relax

\chapter*{Sommario}

Il presente documento descrive il lavoro svolto durante il periodo di stage, della durata di circa trecento ore, dal laureando \myName\ presso l'azienda \myCompany\
Gli obiettivi da raggiungere erano molteplici.

In primo luogo, era previsto lo studio delle tecnologie necessarie per poter approcciare l'implementazione del progetto di stage. Tra queste, sono state centrali
\textit{Java} e \textit{Spring}, con qualche accenno a tecnologie per il front-end, quali \textit{Typescript} e \textit{Angular}.

\noindent È stata dedicata inoltre una parte allo studio dell'architettura a microservizi e a \textit{Spring Cloud}.

In secondo luogo era richiesta l'implementazione del progetto di stage, che è consistito nello sviluppo del back-end di una \textit{web application} per gestire lo \textit{skill matrix}, utilizzato dall'azienda per raccogliere le competenze dei candidati ai colloqui lavorativi.

%\vfill
%
%\selectlanguage{english}
%\pdfbookmark{Abstract}{Abstract}
%\chapter*{Abstract}
%
%\selectlanguage{italian}

\endgroup

\vfill
