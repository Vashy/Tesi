
%**************************************************************
% Acronimi
%**************************************************************
\renewcommand{\acronymname}{Acronimi e abbreviazioni}

\newacronym[description={\glslink{httpg}{HyperText Transfer Protocol}}]
    {http}{HTTP}{HyperText Transfer Protocol}

\newacronym[description={\glslink{apig}{Application Program Interface}}]
    {api}{API}{Application Program Interface}

\newacronym[description={\glslink{umlg}{Unified Modeling Language}}]
    {uml}{UML}{Unified Modeling Language}

\newacronym[description={\glslink{ictg}{Information and Communications Technology}}]
    {ict}{ICT}{Information and Communications Technology}

\newacronym[description={\glslink{isog}{International Organization for Standardization}}]
    {iso}{ISO}{International Organization for Standardization}

\newacronym[description={\glslink{ohsasg}{Occupational Health and Safety Assessment Series}}]
    {ohsas}{OHSAS}{Occupational Health and Safety Assessment Series}

\newacronym{bss}{BSS}{Supporto al Business}

\newacronym{soa}{SOA}{Società Organismi di Attestazione}

\newacronym{eai}{EAI}{Enterprise Application Integration}

\newacronym{it}{IT}{Information Technology}

\newacronym{mit}{MIT}{Massachusetts Institute of Tecnology}

\newacronym[description={\glslink{iotg}{Internet of Things}}]
    {iot}{IoT}{Internet of Things}

\newacronym[description={\glslink{jvmg}{Java Virtual Machine}}]
    {jvm}{JVM}{Java Virtual Machine}

\newacronym[description={\glslink{dbmsg}{Database Management System}}]
    {dbms}{DBMS}{Database Management System}

\newacronym[description={\glslink{jsong}{JavaScript Object Notation}}]
    {json}{JSON}{JavaScript Object Notation}

\newacronym{smtp}{SMTP}{Simple Mail Transfer Protocol}

\newacronym[description={\glslink{restg}{Representational State Transfer}}]
    {rest}{REST}{Representational State Transfer}

%**************************************************************
% Glossario
%**************************************************************

%\renewcommand{\glossaryname}{Glossario}

\newglossaryentry{apig}
{
    name=API,
    text=Application Program Interface,
    sort=api,
    description={In informatica con il termine \emph{Application Programming Interface API} (ing. interfaccia di programmazione di un'applicazione) si indica ogni insieme di procedure disponibili al programmatore, di solito raggruppate a formare un set di strumenti specifici per l'espletamento di un determinato compito all'interno di un certo programma. La finalità è ottenere un'astrazione, di solito tra l'hardware e il programmatore o tra software a basso e quello ad alto livello semplificando così il lavoro di programmazione}
}

\newglossaryentry{umlg}
{
    name=UML,
    text=UML,
    sort=uml,
    description={In ingegneria del software \emph{UML, Unified Modeling Language} (ing. linguaggio di modellazione unificato) è un linguaggio di modellazione e specifica basato sul paradigma object-oriented. L'\emph{UML} svolge un'importantissima funzione di ``lingua franca'' nella comunità della progettazione e programmazione a oggetti. Gran parte della letteratura di settore usa tale linguaggio per descrivere soluzioni analitiche e progettuali in modo sintetico e comprensibile a un vasto pubblico}
}

\newglossaryentry{ictg}
{
    name=Information and Communications Technology,
    text=ICT,
    sort=ict,
    description={in informatica, il termine \emph{Information and Communications Technology} sono l'insieme dei metodi e delle tecniche utilizzate nella trasmissione, ricezione ed elaborazione di dati e informazioni}
}

\newglossaryentry{isog}
{
    name=ISO,
    text=ISO,
    sort=iso,
    description={L'\emph{Organizzazione internazionale per la normazione} è l'organizzazione più importante a livello mondiale per la definizione di \textit{standard}}
}

\newglossaryentry{ohsasg}
{
    name=Occupational Health and Safety Assessment Series,
    text=OHSAS,
    sort=ohsas,
    description={L'\emph{Occupational Health and Safety Assessment Series} identifica uno standard inglese per un sistema di gestione e sicurezza dei lavoratori}
}

\newglossaryentry{big-data}
{
    name=Big Data,
    text=Big Data,
    sort=big data,
    description={Il termine si riferisce a una raccolta di dati (strutturati e non) talmente estesa in termini di velocità, volume e varietà da richiedere apposite tecnologie e metodi per l'estrazione. Ciò che conta non è la quantità di dati, ma come essi vengono utilizzati}
}

\newglossaryentry{iotg}
{
    name=Internet of Things,
    text=IOT,
    sort=iot,
    description={Nelle telecomunicazioni, il termine si riferisce all'estensione di internet al mondo delle cose e luoghi concreti}
}

\newglossaryentry{webapp}
{
    name=Web Application,
    text=Web Application,
    sort=web application,
    description={Un applicazione web è un applicazione distribuita via web, pertanto accessibile per mezzo della rete e di un \emph{browser}}
}

\newglossaryentry{skill-matrix}
{
    name=Skill Matrix,
    text=Skill Matrix,
    sort=skill matrix,
    description={Una matrice (e.g. un foglio di calcolo) usata dall'azienda Sync Lab per documentare le competenze dei candidati interessati all'assunzione, in cui è riportato il nome della competenza, il livello e la categoria. Il documento viene compilato dai candidati prima di un eventuale colloquio}
}

\newglossaryentry{angular}
{
    name=Angular,
    text=Angular,
    sort=angular,
    description={Framework \textit{open source} per lo sviluppo di applicazioni web, rilasciato con licenza \gls{mit}, nato come evoluzione di AngularJS. Il linguaggio di programmazione usato da Angular è TypeScript, a differenza del predecessore che usava \gls{javascript}}
}

\newglossaryentry{spring}
{
    name=Spring,
    text=Spring,
    sort=spring,
    description={Framework \textit{open source} per lo sviluppo di applicazioni con target principale Java, ma supporta ufficialmente anche Kotlin e Groovy. A Spring sono associati molti sotto-progetti, come Spring Boot e Spring Cloud, sviluppati per fornire modularità al framwork}
}

\newglossaryentry{java}
{
    name=Java,
    text=Java,
    sort=java,
    description={Java è un linguaggio di programmazione ad oggetti e a tipizzazione statica \emph{general purpose} nato a metà degli anni 90 che si appoggia alla \gls{jvm}}
}

\newglossaryentry{jvmg}
{
    name=Java Virtual Machine,
    text=Java Virtual Machine,
    sort=jvm,
    description={Componente della piattaforma \gls{java} che esegue programmi tradotti in \emph{bytecode} dopo una fase di compilazione}
}

\newglossaryentry{dbmsg}
{
    name=Database Management System,
    text=DBMS,
    sort=dbms,
    description={In informatica, il termine \gls{dbms} indica un sistema software progettato per consentire la manipolazione, la creazione e l'interrogazione efficiente su un database}
}

\newglossaryentry{mongodb}
{
    name=MongoDB,
    text=MongoDB,
    sort=MongoDB,
    description={È un \gls{dbms} non relazionale (NoSQL) orientato ai documenti: ciò significa che abbandona la struttura tradizionale basata su tabelle dei database relazionali in favore di documenti in stile \gls{json}}
}

\newglossaryentry{jsong}
{
    name=JSON,
    text=JSON,
    sort=JSON,
    description={Acronimo di JavaScript Object Notation, è un formato usato nel web adatto allo scambio dei dati fra applicazioni client-server. Si basa sul linguaggio \gls{javascript} da cui prende spunto per la sintassi}
}

\newglossaryentry{cloud-computing}
{
    name=Cloud Computing,
    text=Cloud Computing,
    sort=Cloud Computing,
    description={È un paradigma di erogazione di servizi \emph{on demand} offerti ad un cliente da un fornitore attraverso il web, a partire da
    un sistema con risorse configurabili e preesistenti, generalmente disponibili in remoto}
}

\newglossaryentry{javascript}
{
    name=JavaScript,
    text=JavaScript,
    sort=JavaScript,
    description={Linguaggio di scripting orientato agli oggetti ed eventi comunemente utilizzato nella programmazione web lato client, anche se è stato recentemente esteso anche al lato server}
}

\newglossaryentry{oracle}
{
    name=Database Oracle,
    text=Database Oracle,
    sort=Database Oracle,
    description={È uno dei più noti \gls{dbms} relazionali, è scritto in linguaggio C. È prodotto dalla \emph{Oracle Corporation}}
}

\newglossaryentry{httpg}
{
    name=HTTP,
    text=HTTP,
    sort=HTTP,
    description={In telecomunicazioni, l'HyperText Transfer Protocol è un protocollo usato come sistema per la trasmissione di informazioni sul web}
}

\newglossaryentry{microservizio}
{
    name=Microservizio,
    text=microservizio,
    sort=Microservizio,
    description={Componente di un'architetettura a microservizi, variante dell'architettura orientata ai servizi, in cui l'applicazione è strutturata come un insieme di servizi scarsamente accoppiati}
}

\newglossaryentry{restg}
{
    name=REST,
    text=REST,
    sort=REST,
    description={Representational State Transfer è uno stile architetturale usato nei sistemi distribuiti che si riferisce a un sistema di trasmissione dati via \acrshort{http} senza \emph{layer} addizionali. I sistemi REST infatti non prevedono il concetto di sessione, essendo \textit{stateless}}
}

\newglossaryentry{spring-cloud}
{
    name=Spring Cloud,
    text=Spring Cloud,
    sort=Spring Cloud,
    description={\textit{Framework} del team Pivotal, sotto-progetto di \gls{spring}, che offre gli strumenti necessari agli sviluppatori per implementare alcuni dei \textit{pattern} più comuni per i sistemi distribuiti}
}

\newglossaryentry{typescript}
{
    name=TypeScript,
    text=TypeScript,
    sort=TypeScript,
    description={Linguaggio di programmazione \emph{open source} sviluppato da \emph{Microsoft}, super-set di \gls{javascript}, che estende la sintassi rendendolo un linguaggio a tipizzazione forte}
}