
%**************************************************************
% Acronimi
%**************************************************************
\renewcommand{\acronymname}{Acronimi e abbreviazioni}

\newacronym[description={\glslink{apig}{Application Program Interface}}]
    {api}{API}{Application Program Interface}

\newacronym[description={\glslink{umlg}{Unified Modeling Language}}]
    {uml}{UML}{Unified Modeling Language}

\newacronym[description={\glslink{ictg}{Information and Communications Technology}}]
    {ict}{ICT}{Information and Communications Technology}

\newacronym[description={\glslink{isog}{International Organization for Standardization}}]
    {iso}{ISO}{International Organization for Standardization}

\newacronym[description={\glslink{ohsasg}{Occupational Health and Safety Assessment Series}}]
    {ohsas}{OHSAS}{Occupational Health and Safety Assessment Series}

\newacronym{bss}{BSS}{Supporto al Business}

\newacronym{soa}{SOA}{Società Organismi di Attestazione}

\newacronym{eai}{EAI}{Enterprise Application Integration}

\newacronym{it}{IT}{Information Technology}

\newacronym[description={\glslink{iotg}{Internet of Things}}]
    {iot}{IoT}{Internet of Things}

%**************************************************************
% Glossario
%**************************************************************
%\renewcommand{\glossaryname}{Glossario}

\newglossaryentry{apig}
{
    name=\glslink{api}{API},
    text=Application Program Interface,
    sort=api,
    description={in informatica con il termine \emph{Application Programming Interface API} (ing. interfaccia di programmazione di un'applicazione) si indica ogni insieme di procedure disponibili al programmatore, di solito raggruppate a formare un set di strumenti specifici per l'espletamento di un determinato compito all'interno di un certo programma. La finalità è ottenere un'astrazione, di solito tra l'hardware e il programmatore o tra software a basso e quello ad alto livello semplificando così il lavoro di programmazione}
}

\newglossaryentry{umlg}
{
    name=\glslink{uml}{UML},
    text=UML,
    sort=uml,
    description={in ingegneria del software \emph{UML, Unified Modeling Language} (ing. linguaggio di modellazione unificato) è un linguaggio di modellazione e specifica basato sul paradigma object-oriented. L'\emph{UML} svolge un'importantissima funzione di ``lingua franca'' nella comunità della progettazione e programmazione a oggetti. Gran parte della letteratura di settore usa tale linguaggio per descrivere soluzioni analitiche e progettuali in modo sintetico e comprensibile a un vasto pubblico}
}

\newglossaryentry{ictg}
{
    name=\glslink{ict}{ICT},
    text=ICT,
    sort=ict,
    description={in informatica, il termine \emph{Information and Communications Technology} sono l'insieme dei metodi e delle tecniche utilizzate nella trasmissione, ricezione ed elaborazione di dati e informazioni.}
}

\newglossaryentry{isog}
{
    name=\glslink{iso}{ISO},
    text=ISO,
    sort=iso,
    description={L'\emph{Organizzazione internazionale per la normazione} è l'organizzazione più importante a livello mondiale per la definizione di \textit{standard}.}
}

\newglossaryentry{ohsasg}
{
    name=\glslink{ohsas}{OHSAS},
    text=OHSAS,
    sort=ohsas,
    description={L'\emph{Occupational Health and Safety Assessment Series} identifica uno standard inglese per un sistema di gestione e sicurezza dei lavoratori.}
}

\newglossaryentry{big-data}
{
    name=Big Data,
    text=Big Data,
    sort=big data,
    description={Il termine si riferisce a una raccolta di dati (strutturati e non) talmente estesa in termini di velocità, volume e varietà da richiedere apposite tecnologie e metodi per l'estrazione. Ciò che conta non è la quantità di dati, ma come essi vengono utilizzati.}
}

\newglossaryentry{iotg}
{
    name=\glslink{iot}{IOT},
    text=IOT,
    sort=iot,
    description={Nelle telecomunicazioni, il termine si riferisce all'estensione di internet al mondo delle cose e luoghi concreti.}
}
