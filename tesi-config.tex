%**************************************************************
% file contenente le impostazioni della tesi
%**************************************************************

%**************************************************************
% Frontespizio
%**************************************************************

% Autore
\newcommand{\myName}{Timoty Granziero}
\newcommand{\myTitle}{Approccio a Microservizi nello sviluppo di Web Application moderne}
% Tipo di tesi
\newcommand{\myDegree}{Tesi di laurea triennale}

% Università
\newcommand{\myUni}{Università degli Studi di Padova}

% Facoltà
\newcommand{\myFaculty}{Corso di Laurea in Informatica}

% Dipartimento
\newcommand{\myDepartment}{Dipartimento di Matematica "Tullio Levi-Civita"}

% Titolo del relatore
\newcommand{\profTitle}{Prof.}

% Relatore
\newcommand{\myProf}{Gilberto File'}

% Luogo
\newcommand{\myLocation}{Padova}

% Anno accademico
\newcommand{\myAA}{2018-2019}

% Data discussione
\newcommand{\myTime}{18 Luglio 2019}

\newcommand{\myCompany}{Sync Lab s.r.l.}

\newcommand{\fabio}{Fabio Pallaro}


%**************************************************************
% Impostazioni di impaginazione
% see: http://wwwcdf.pd.infn.it/AppuntiLinux/a2547.htm
%**************************************************************

\setlength{\parindent}{0pt}   % larghezza rientro della prima riga
\setlength{\parskip}{3pt}   % distanza tra i paragrafi


%**************************************************************
% Impostazioni di biblatex
%**************************************************************

\bibliography{bibliografia} % database di biblatex

\defbibheading{bibliography} {
    \cleardoublepage
    \phantomsection
    \addcontentsline{toc}{chapter}{\bibname}
    \chapter*{\bibname\markboth{\bibname}{\bibname}}
}

\setlength\bibitemsep{1.5\itemsep} % spazio tra entry

\DeclareBibliographyCategory{opere}
\DeclareBibliographyCategory{web}

% \addtocategory{opere}{womak:lean-thinking}
% \addtocategory{web}{site:agile-manifesto}

\defbibheading{opere}{\section*{Riferimenti bibliografici}}
\defbibheading{web}{\section*{Siti Web consultati}}

\glossarystyle{listgroup}


%**************************************************************
% Impostazioni di caption
%**************************************************************
\captionsetup{
    tableposition=top,
    figureposition=bottom,
    font=small,
    format=hang,
    labelfont=bf
}

%**************************************************************
% Impostazioni di glossaries
%**************************************************************

%**************************************************************
% Acronimi
%**************************************************************
\renewcommand{\acronymname}{Acronimi e abbreviazioni}

\newacronym[description={\glslink{apig}{Application Program Interface}}]
    {api}{API}{Application Program Interface}

\newacronym[description={\glslink{umlg}{Unified Modeling Language}}]
    {uml}{UML}{Unified Modeling Language}

\newacronym[description={\glslink{ictg}{Information and Communications Technology}}]
    {ict}{ICT}{Information and Communications Technology}

\newacronym[description={\glslink{isog}{International Organization for Standardization}}]
    {iso}{ISO}{International Organization for Standardization}

\newacronym[description={\glslink{ohsasg}{Occupational Health and Safety Assessment Series}}]
    {ohsas}{OHSAS}{Occupational Health and Safety Assessment Series}

\newacronym{bss}{BSS}{Supporto al Business}

\newacronym{soa}{SOA}{Società Organismi di Attestazione}

\newacronym{eai}{EAI}{Enterprise Application Integration}

\newacronym{it}{IT}{Information Technology}

\newacronym[description={\glslink{iotg}{Internet of Things}}]
    {iot}{IoT}{Internet of Things}

%**************************************************************
% Glossario
%**************************************************************
%\renewcommand{\glossaryname}{Glossario}

\newglossaryentry{apig}
{
    name=\glslink{api}{API},
    text=Application Program Interface,
    sort=api,
    description={in informatica con il termine \emph{Application Programming Interface API} (ing. interfaccia di programmazione di un'applicazione) si indica ogni insieme di procedure disponibili al programmatore, di solito raggruppate a formare un set di strumenti specifici per l'espletamento di un determinato compito all'interno di un certo programma. La finalità è ottenere un'astrazione, di solito tra l'hardware e il programmatore o tra software a basso e quello ad alto livello semplificando così il lavoro di programmazione}
}

\newglossaryentry{umlg}
{
    name=\glslink{uml}{UML},
    text=UML,
    sort=uml,
    description={in ingegneria del software \emph{UML, Unified Modeling Language} (ing. linguaggio di modellazione unificato) è un linguaggio di modellazione e specifica basato sul paradigma object-oriented. L'\emph{UML} svolge un'importantissima funzione di ``lingua franca'' nella comunità della progettazione e programmazione a oggetti. Gran parte della letteratura di settore usa tale linguaggio per descrivere soluzioni analitiche e progettuali in modo sintetico e comprensibile a un vasto pubblico}
}

\newglossaryentry{ictg}
{
    name=\glslink{ict}{ICT},
    text=ICT,
    sort=ict,
    description={in informatica, il termine \emph{Information and Communications Technology} sono l'insieme dei metodi e delle tecniche utilizzate nella trasmissione, ricezione ed elaborazione di dati e informazioni.}
}

\newglossaryentry{isog}
{
    name=\glslink{iso}{ISO},
    text=ISO,
    sort=iso,
    description={L'\emph{Organizzazione internazionale per la normazione} è l'organizzazione più importante a livello mondiale per la definizione di \textit{standard}.}
}

\newglossaryentry{ohsasg}
{
    name=\glslink{ohsas}{OHSAS},
    text=OHSAS,
    sort=ohsas,
    description={L'\emph{Occupational Health and Safety Assessment Series} identifica uno standard inglese per un sistema di gestione e sicurezza dei lavoratori.}
}

\newglossaryentry{big-data}
{
    name=Big Data,
    text=Big Data,
    sort=big data,
    description={Il termine si riferisce a una raccolta di dati (strutturati e non) talmente estesa in termini di velocità, volume e varietà da richiedere apposite tecnologie e metodi per l'estrazione. Ciò che conta non è la quantità di dati, ma come essi vengono utilizzati.}
}

\newglossaryentry{iotg}
{
    name=\glslink{iot}{IOT},
    text=IOT,
    sort=iot,
    description={Nelle telecomunicazioni, il termine si riferisce all'estensione di internet al mondo delle cose e luoghi concreti.}
}
 % database di termini
\makeglossaries


%**************************************************************
% Impostazioni di graphicx
%**************************************************************
\graphicspath{{immagini/}} % cartella dove sono riposte le immagini


%**************************************************************
% Impostazioni di hyperref
%**************************************************************
\hypersetup{
    %hyperfootnotes=false,
    %pdfpagelabels,
    %draft,	% = elimina tutti i link (utile per stampe in bianco e nero)
    colorlinks=true,
	linktocpage=true,
    pdfstartpage=1,
    pdfstartview=FitV,
    % decommenta la riga seguente per avere link in nero (per esempio per la stampa in bianco e nero)
    %colorlinks=false, linktocpage=false, pdfborder={0 0 0}, pdfstartpage=1, pdfstartview=FitV,
    breaklinks=true,
    pdfpagemode=UseNone,
    pageanchor=true,
    pdfpagemode=UseOutlines,
    plainpages=false,
    bookmarksnumbered,
    bookmarksopen=true,
    bookmarksopenlevel=1,
    hypertexnames=true,
    pdfhighlight=/O,
    %nesting=true,
    %frenchlinks,
    urlcolor=webbrown,
    linkcolor=black,
    citecolor=webgreen,
    %pagecolor=RoyalBlue,
    %urlcolor=Black, linkcolor=Black, citecolor=Black, %pagecolor=Black,
    pdftitle={\myTitle},
    pdfauthor={\textcopyright\ \myName, \myUni, \myFaculty},
    pdfsubject={},
    pdfkeywords={},
    pdfcreator={pdfLaTeX},
    pdfproducer={LaTeX}
}

%**************************************************************
% Impostazioni di itemize
%**************************************************************
%\renewcommand{\labelitemi}{$\ast$}

\renewcommand{\labelitemi}{$\cdot$}
\renewcommand{\labelitemii}{$\circ$}
%\renewcommand{\labelitemiii}{$\diamond$}
%\renewcommand{\labelitemiv}{$\ast$}


%**************************************************************
% Impostazioni di listings
%**************************************************************
\lstset{
    language=Java,
    keywordstyle=\color{BurntOrange}\bfseries,
    basicstyle=\ttfamily,
%    identifierstyle=\color{Green},
    commentstyle=\color{Bittersweet}\ttfamily,
    stringstyle=\rmfamily\color{Green},
    numbers=none, %left,%
    numberstyle=\scriptsize, %\tiny
    stepnumber=5,
    numbersep=8pt,
    showstringspaces=false,
    breaklines=true,
    frameround=ftff,
    frame=none,
}


%**************************************************************
% Impostazioni di xcolor
%**************************************************************

\definecolor{webgreen}{rgb}{0,.5,0}
\definecolor{webbrown}{rgb}{.6,0,0}


%**************************************************************
% Altro
%**************************************************************

\newcommand{\omissis}{[\dots\negthinspace]} % produce [...]

% eccezioni all'algoritmo di sillabazione
\hyphenation
{
    ma-cro-istru-zio-ne
    gi-ral-din
}

\newcommand{\sectionname}{sezione}
\addto\captionsitalian{\renewcommand{\figurename}{Figura}
                       \renewcommand{\tablename}{Tabella}}

% \newcommand{\glsfirstoccur}{\ap{{[g]}}}
%\newcommand{\gloss}[1]{\textit{#1}\ped{\tiny{G}}}

\newcommand{\gloss}{\ped{\tiny{G}}}

\newcommand{\intro}[1]{\emph{\textsf{#1}}}

%**************************************************************
% Environment per ``rischi''
%**************************************************************
\newcounter{riskcounter}                % define a counter
\setcounter{riskcounter}{0}             % set the counter to some initial value

%%%% Parameters
% #1: Title
\newenvironment{risk}[1]{
    \refstepcounter{riskcounter}        % increment counter
    \par \noindent                      % start new paragraph
    \textbf{\arabic{riskcounter}. #1}   % display the title before the
                                        % content of the environment is displayed
}{
    \par\medskip
}

\newcommand{\riskname}{Rischio}

\newcommand{\riskdescription}[1]{\textbf{\\Descrizione:} #1.}

\newcommand{\risksolution}[1]{\textbf{\\Soluzione:} #1.}


%********************
% * Indici personalizzati *
%********************

% Indice per Code Snippets
\makeatletter
\newcommand\listcodesname{Elenco degli snippet}
\newcommand\listofcodes{%
	\chapter*{\listcodesname}\@starttoc{codes}}
\makeatother

% Indice per Use cases
\makeatletter
\newcommand\listucname{Elenco degli Use Case}
\newcommand\listofuc{%
	\chapter*{\listucname}\@starttoc{uc}}
\makeatother


% ****************************
% * Environment per ``use case''   *
% ****************************

% new tcolorbox environment for use cases
% #1: tcolorbox options
% #2: color
% #3: box title
\newtcolorbox{usecasebox}[3][]
{
	colframe = #2!25,
	colback  = #2!10,
	coltitle = #2!20!black,
	title    = {\bfseries #3},
	parbox=false,
}

\newcounter{usecasecounter}             % define a counter
\setcounter{usecasecounter}{0}          % set the counter to some initial value

%%%% Parameters
% #1: ID
% #2: Servizio
% #3: Nome
\newenvironment{usecase}[3]{
	\medskip

	\setlength{\parskip}{0pt}
	\newcommand{\usecasetitle}{\usecasename #1-#2: #3}
	\usecasebox{lightgray}{\large\textbf{\usecasetitle}}
	\renewcommand{\theusecasecounter}{\usecasename #1-#2}
 	\refstepcounter{usecasecounter}             % increment counter
	\addcontentsline{uc}{section}{\usecasetitle}
}{\endusecasebox%
	\medskip%
}

\newcommand{\usecasename}{UC}

\newcommand{\usecaseactors}[1]{\textbf{\\Attori Principali:} #1.}% \vspace{4pt}}
\newcommand{\usecasepre}[1]{\textbf{\\Precondizioni:} #1.}% \vspace{4pt}}
\newcommand{\usecasedesc}[1]{\textbf{\\Descrizione:} #1.}% \vspace{4pt}}
\newcommand{\usecasepost}[1]{\textbf{\\Postcondizioni:} #1.}% \vspace{4pt}}
\newcommand{\usecasealt}[1]{\textbf{\\Scenario Alternativo:} #1.}% \vspace{4pt}}

\newenvironment{ucitemize}%
{\begin{itemize}[topsep=0pt, partopsep=0pt]%
		\setlength{\itemsep}{2pt}%
		\setlength{\parskip}{0pt}}%
	{\end{itemize}}

\newenvironment{ucenumerate}%
{\enumerate[topsep=3pt, partopsep=0pt]%
		\setlength{\itemsep}{2pt}%
		\setlength{\parskip}{0pt}}%
	{\endenumerate}


\newenvironment{ucestensioni}[0]{%
	\vspace{4pt}%

	\textbf{Estensioni:}
	\ucenumerate%
}{\enducenumerate}

\newenvironment{ucgeneralizzazioni}{
	\vspace{4pt}%

	\textbf{Specializzazioni:}%
	\ucenumerate%
}{\enducenumerate}

\newenvironment{ucscenarioprincipale}{
	\vspace{4pt}
	
	\textbf{Scenario principale:}
	\ucenumerate
}{
	\enducenumerate
}

%**************************************************************
% Requisiti
%**************************************************************

\newcounter{reqcounter}
%\setcounter{reqcounter}{0}
\newcounter{subreqcounter}[reqcounter]

\newcommand{\reqrow}[3]{%
	\stepcounter{reqcounter}%
	#1-\thereqcounter & #2 & #3\\%
}

\newcommand{\reqtwocol}[2]{%
	\stepcounter{reqcounter}%
	#1-\thereqcounter & #2\\%
}

\newcommand{\subreqtwocol}[2]{%
	\stepcounter{subreqcounter}%
	#1-\thereqcounter.\thesubreqcounter & #2\\%
}


%**************************************************************
% Environment per ``namespace description''
%**************************************************************

\newenvironment{namespacedesc}{
    \vspace{10pt}
    \par \noindent                              % start new paragraph
    \begin{description}
}{
    \end{description}
    \medskip
}

\newcommand{\classdesc}[2]{\item[\textbf{#1:}] #2}


% subsubsection 4
% paragraph 5
% ecc..
\setcounter{secnumdepth}{4}
\setcounter{tocdepth}{4}


% Per tabularx con padding, parametro tra [], eg [1.3]
\newenvironment{paddedtablex}[1][1]{%
	\renewcommand*{\arraystretch}{#1}%
%	\renewcommand\theadfont{\bfseries}%
	\tabularx%
}{\endtabularx}
