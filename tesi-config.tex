%**************************************************************
% file contenente le impostazioni della tesi
%**************************************************************

%**************************************************************
% Frontespizio
%**************************************************************

% Autore
\newcommand{\myName}{Timoty Granziero}
\newcommand{\myTitle}{Approccio a Microservizi nello sviluppo di Web Application moderne}
% Tipo di tesi
\newcommand{\myDegree}{Tesi di laurea triennale}

% Università
\newcommand{\myUni}{Università degli Studi di Padova}

% Facoltà
\newcommand{\myFaculty}{Corso di Laurea in Informatica}

% Dipartimento
\newcommand{\myDepartment}{Dipartimento di Matematica "Tullio Levi-Civita"}

% Titolo del relatore
\newcommand{\profTitle}{Prof.}

% Relatore
\newcommand{\myProf}{Gilberto File'}

% Luogo
\newcommand{\myLocation}{Padova}

% Anno accademico
\newcommand{\myAA}{2018-2019}

% Data discussione
\newcommand{\myTime}{18 Luglio 2019}

\newcommand{\myCompany}{Sync Lab s.r.l.}

\newcommand{\fabio}{Fabio Pallaro}


%**************************************************************
% Impostazioni di impaginazione
% see: http://wwwcdf.pd.infn.it/AppuntiLinux/a2547.htm
%**************************************************************

\setlength{\parindent}{0pt}   % larghezza rientro della prima riga
\setlength{\parskip}{3pt}   % distanza tra i paragrafi


%**************************************************************
% Impostazioni di biblatex
%**************************************************************

\bibliography{bibliografia} % database di biblatex

\defbibheading{bibliography} {
    \cleardoublepage
    \phantomsection
    \addcontentsline{toc}{chapter}{\bibname}
    \chapter*{\bibname\markboth{\bibname}{\bibname}}
}

\setlength\bibitemsep{1.5\itemsep} % spazio tra entry

\DeclareBibliographyCategory{opere}
\DeclareBibliographyCategory{web}

% \addtocategory{opere}{womak:lean-thinking}
% \addtocategory{web}{site:agile-manifesto}

\defbibheading{opere}{\section*{Riferimenti bibliografici}}
\defbibheading{web}{\section*{Siti Web consultati}}

\glossarystyle{listgroup}


%**************************************************************
% Impostazioni di caption
%**************************************************************
\captionsetup{
    tableposition=top,
    figureposition=bottom,
    font=small,
    format=hang,
    labelfont=bf
}

%**************************************************************
% Impostazioni di glossaries
%**************************************************************

%**************************************************************
% Acronimi
%**************************************************************
%\renewcommand{\acronymname}{Acronimi e abbreviazioni}

\newacronym[description={\glslink{httpg}{HyperText Transfer Protocol}}]
    {http}{HTTP}{HyperText Transfer Protocol}

\newacronym[description={\glslink{apig}{Application Program Interface}}]
    {api}{API}{Application Program Interface}

\newacronym[description={\glslink{umlg}{Unified Modeling Language}}]
    {uml}{UML}{Unified Modeling Language}

\newacronym[description={\glslink{ictg}{Information and Communications Technology}}]
    {ict}{ICT}{Information and Communications Technology}

\newacronym[description={\glslink{isog}{International Organization for Standardization}}]
    {iso}{ISO}{International Organization for Standardization}

\newacronym[description={\glslink{ohsasg}{Occupational Health and Safety Assessment Series}}]
    {ohsas}{OHSAS}{Occupational Health and Safety Assessment Series}

\newacronym{bss}{BSS}{Supporto al Business}

\newacronym{soa}{SOA}{Società Organismi di Attestazione}

\newacronym{eai}{EAI}{Enterprise Application Integration}

\newacronym{it}{IT}{Information Technology}

\newacronym{mit}{MIT}{Massachusetts Institute of Tecnology}

\newacronym[description={\glslink{iotg}{Internet of Things}}]
    {iot}{IoT}{Internet of Things}

\newacronym[description={\glslink{jvmg}{Java Virtual Machine}}]
    {jvm}{JVM}{Java Virtual Machine}

\newacronym[description={\glslink{dbmsg}{Database Management System}}]
    {dbms}{DBMS}{Database Management System}

\newacronym[description={\glslink{jsong}{JavaScript Object Notation}}]
    {json}{JSON}{JavaScript Object Notation}

\newacronym{smtp}{SMTP}{Simple Mail Transfer Protocol}

\newacronym[description={\glslink{restg}{Representational State Transfer}}]
    {rest}{REST}{Representational State Transfer}

\newacronym{ieee}{IEEE}{Institute of Electrical and Electronics Engineers}

\newacronym{ui}{UI}{User Interface}

\newacronym{ide}{IDE}{Integrated Development Environment}

\newacronym{ux}{UX}{User Experience}

\newacronym{html}{HTML}{HyperText Markup Language}

\newacronym{css}{CSS}{Cascading Style Sheets}

\newacronym{ram}{RAM}{Random Access Memory}

\newacronym{cli}{CLI}{Command Line Interface}

\newacronym{vcs}{VCS}{Version Control System}

\newacronym{dql}{DQL}{Database Query Language}

\newacronym{jdk}{JDK}{Java Development Kit}

\newacronym{cfu}{CFU}{Credito Formativo Universitario}

\newacronym{ioc}{IoC}{Inversion of Control}

%**************************************************************
% Glossario
%**************************************************************

%\renewcommand{\glossaryname}{Glossario}

\newglossaryentry{apig}
{
    name=API,
    text=Application Program Interface,
    sort=api,
    description={In informatica con il termine \emph{Application Programming Interface API} (ing. interfaccia di programmazione di un'applicazione) si indica ogni insieme di procedure disponibili al programmatore, di solito raggruppate a formare un set di strumenti specifici per l'espletamento di un determinato compito all'interno di un certo programma. La finalità è ottenere un'astrazione, di solito tra l'hardware e il programmatore o tra software a basso e quello ad alto livello semplificando così il lavoro di programmazione}
}

\newglossaryentry{umlg}
{
    name=UML,
    text=UML,
    sort=uml,
    description={In ingegneria del software \emph{UML, Unified Modeling Language} (ing. linguaggio di modellazione unificato) è un linguaggio di modellazione e specifica basato sul paradigma object-oriented. L'\emph{UML} svolge un'importantissima funzione di ``lingua franca'' nella comunità della progettazione e programmazione a oggetti. Gran parte della letteratura di settore usa tale linguaggio per descrivere soluzioni analitiche e progettuali in modo sintetico e comprensibile a un vasto pubblico}
}

\newglossaryentry{ictg}
{
    name=Information and Communications Technology,
    text=ICT,
    sort=ict,
    description={in informatica, il termine \emph{Information and Communications Technology} sono l'insieme dei metodi e delle tecniche utilizzate nella trasmissione, ricezione ed elaborazione di dati e informazioni}
}

\newglossaryentry{isog}
{
    name=ISO,
    text=ISO,
    sort=iso,
    description={L'\emph{Organizzazione internazionale per la normazione} è l'organizzazione più importante a livello mondiale per la definizione di \textit{standard}}
}

\newglossaryentry{ohsasg}
{
    name=Occupational Health and Safety Assessment Series,
    text=OHSAS,
    sort=ohsas,
    description={L'\emph{Occupational Health and Safety Assessment Series} identifica uno standard inglese per un sistema di gestione e sicurezza dei lavoratori}
}

\newglossaryentry{big-data}
{
    name=Big Data,
    text=Big Data,
    sort=big data,
    description={Il termine si riferisce a una raccolta di dati (strutturati e non) talmente estesa in termini di velocità, volume e varietà da richiedere apposite tecnologie e metodi per l'estrazione. Ciò che conta non è la quantità di dati, ma come essi vengono utilizzati}
}

\newglossaryentry{iotg}
{
    name=Internet of Things,
    text=IOT,
    sort=iot,
    description={Nelle telecomunicazioni, il termine si riferisce all'estensione di internet al mondo delle cose e luoghi concreti}
}

\newglossaryentry{webapp}
{
    name=Web Application,
    text=web application,
    sort=web application,
    description={Un applicazione web è un applicazione distribuita via web, pertanto accessibile per mezzo della rete e di un \emph{browser}}
}

\newglossaryentry{skill-matrix}
{
    name=Skill Matrix,
    text=skill matrix,
    sort=skill matrix,
    description={Una matrice (e.g. un foglio di calcolo) usata dall'azienda Sync Lab per documentare le competenze dei candidati interessati all'assunzione, in cui è riportato il nome della competenza, il livello e la categoria. Il documento viene compilato dai candidati prima di un eventuale colloquio}
}

\newglossaryentry{angular}
{
    name=Angular,
    text=Angular,
    sort=angular,
    description={Framework \textit{open source} per lo sviluppo di applicazioni web, rilasciato con licenza \gls{mit}, nato come evoluzione di AngularJS. Il linguaggio di programmazione usato da Angular è \gls{typescript}, a differenza del predecessore che usava \gls{javascript}}
}

\newglossaryentry{spring}
{
    name=Spring,
    text=Spring,
    sort=spring,
    description={Framework \textit{open source} per lo sviluppo di applicazioni con target principale Java, ma supporta ufficialmente anche Kotlin e Groovy. A Spring sono associati molti sotto-progetti, come Spring Boot e Spring Cloud, sviluppati per fornire modularità al framwork}
}

\newglossaryentry{java}
{
    name=Java,
    text=Java,
    sort=java,
    description={Java è un linguaggio di programmazione ad oggetti e a tipizzazione statica \emph{general purpose} nato a metà degli anni 90 che si appoggia alla \gls{jvm}}
}

\newglossaryentry{jvmg}
{
    name=Java Virtual Machine,
    text=Java Virtual Machine,
    sort=jvm,
    description={Componente della piattaforma \gls{java} che esegue programmi tradotti in \emph{bytecode} dopo una fase di compilazione}
}

\newglossaryentry{dbmsg}
{
    name=Database Management System,
    text=DBMS,
    sort=dbms,
    description={In informatica, il termine \gls{dbms} indica un sistema software progettato per consentire la manipolazione, la creazione e l'interrogazione efficiente su un database}
}

\newglossaryentry{mongodb}
{
    name=MongoDB,
    text=MongoDB,
    sort=MongoDB,
    description={È un \gls{dbms} non relazionale (NoSQL) orientato ai documenti: ciò significa che abbandona la struttura tradizionale basata su tabelle dei database relazionali in favore di documenti in stile \gls{json}}
}

\newglossaryentry{jsong}
{
    name=JSON,
    text=JSON,
    sort=JSON,
    description={Acronimo di JavaScript Object Notation, è un formato usato nel web adatto allo scambio dei dati fra applicazioni client-server. Si basa sul linguaggio \gls{javascript} da cui prende spunto per la sintassi}
}

\newglossaryentry{cloud-computing}
{
    name=Cloud Computing,
    text=Cloud Computing,
    sort=Cloud Computing,
    description={È un paradigma di erogazione di servizi \emph{on demand} offerti ad un cliente da un fornitore attraverso il web, a partire da
    un sistema con risorse configurabili e preesistenti, generalmente disponibili in remoto}
}

\newglossaryentry{javascript}
{
    name=JavaScript,
    text=JavaScript,
    sort=JavaScript,
    description={Linguaggio di scripting orientato agli oggetti ed eventi comunemente utilizzato nella programmazione web lato client, anche se è stato recentemente esteso anche al lato server}
}

\newglossaryentry{oracle}
{
    name=Database Oracle,
    text=Database Oracle,
    sort=Database Oracle,
    description={È uno dei più noti \gls{dbms} relazionali, è scritto in linguaggio C. È prodotto dalla \emph{Oracle Corporation}}
}

\newglossaryentry{httpg}
{
    name=HTTP,
    text=HTTP,
    sort=HTTP,
    description={In telecomunicazioni, l'HyperText Transfer Protocol è un protocollo usato come sistema per la trasmissione di informazioni sul web}
}

\newglossaryentry{microservizio}
{
    name=Microservizio,
    text=microservizio,
    sort=Microservizio,
    description={Componente di un'architetettura a microservizi, variante dell'architettura orientata ai servizi, in cui l'applicazione è strutturata come un insieme di servizi scarsamente accoppiati}
}

\newglossaryentry{restg}
{
    name=REST,
    text=REST,
    sort=REST,
    description={Representational State Transfer è uno stile architetturale usato nei sistemi distribuiti che si riferisce a un sistema di trasmissione dati via \acrshort{http} senza \emph{layer} addizionali. I sistemi REST infatti non prevedono il concetto di sessione, essendo \textit{stateless}}
}

\newglossaryentry{spring-cloud}
{
    name=Spring Cloud,
    text=Spring Cloud,
    sort=Spring Cloud,
    description={\textit{Framework} del team Pivotal, sotto-progetto di \gls{spring}, che offre gli strumenti necessari agli sviluppatori per implementare alcuni dei \textit{pattern} più comuni per i sistemi distribuiti}
}

\newglossaryentry{typescript}
{
    name=TypeScript,
    text=TypeScript,
    sort=TypeScript,
    description={Linguaggio di programmazione \emph{open source} sviluppato da \emph{Microsoft}, super-set di \gls{javascript}, che estende la sintassi rendendolo un linguaggio tipizzato}
}

\newglossaryentry{deploy}
{
    name=Deploy,
    text=deploy,
    sort=Deploy,
    description={Nell'informatica, è un termine che si riferisce al \emph{deployment}, ossia la consegna o rilascio al cliente finale con eventuale installazione e messa in funzione di un sistema software}
}


\newglossaryentry{mocking}
{
    name=Mocking,
    text=mocking,
    sort=Mocking,
    description={Nell'informatica, i \emph{mock object} sono oggetti che simulano e riproducono il comportamento degli oggetti reali. Vengono usati per testare il comportamento di altri oggetti legati a un oggetto talvolta inaccessibile o non implementato}
}

\newglossaryentry{srp}
{
    name=Single Responsibility Principle,
    text=Single Responsibility Principle,
    sort=Single Responsibility Principle,
    description={Nella programmazione orientata agli oggetti questo principio afferma che ogni elemento di un sistema (variabile, classe, metodo) deve avere una e una sola responsabilità, e che tale responsabilità deve essere incapsulata dall'elemento stesso}
}

\newglossaryentry{design-pattern}
{
    name=Design Pattern,
    text=design pattern,
    sort=Design Pattern,
    description={Nell'ingegneria del software, un \emph{design pattern} è una soluzione progettuale generale ad un problema ricorrente. Si tratta, in pratica, di un modello astratto da applicare per la risoluzione di un problema che può presentarsi in diverse situazioni durante le fasi di progettazione e sviluppo di un sistema software}
}

\newglossaryentry{information-hiding}
{
    name=Information Hiding,
    text=information hiding,
    sort=Information Hiding,
    description={Principio della programmazione ad oggetti che si riferisce all'uso dell'incapsulamento per limitare l'accesso diretto agli elementi di un oggetto, per evitare danni collaterali nell'uso di dati inconsistenti con la logica prevista del programma}
}

\newglossaryentry{container}
{
    name=Container,
    text=container,
    sort=Container,
    description={Nell'informatica, un \emph{container} è un'istanza isolata nello spazio utente di un componente o sistema}
}

\newglossaryentry{macchina-virtuale}
{
    name=Macchina Virtuale,
    text=macchina virtuale,
    sort=Macchina Virtuale,
    description={In informatica, con macchina virtuale ci si riferisce a un software che crea un ambiente virtuale che emula tipicamente il comportamento di una macchina fisica tramite un processo di virtualizzazione, grazie all'assegnazione di risorse \emph{hardware}, quali \gls{ram}, porzioni di disco rigido, etc\dots }
}

\newglossaryentry{snippet}
{
    name=Snippet,
    text=snippet,
    sort=Snippet,
    description={Frammento di codice sorgente, generalmente distribuiti nel pubblico dominio}
}

\newglossaryentry{git}
{
    name=Git,
    text=Git,
    sort=Git,
    description={Sistema di controllo versione distribuito, utilizzabile da \gls{cli}, creato da Linus Torvalds nel 2005}
}

\newglossaryentry{repository}
{
    name=Repository,
    text=repository,
    sort=Repository,
    description={Ambiente di un sistema informatico in cui vengono gestiti metadati attraverso tabelle relazionali. Può essere implementato attraverso numerosi sistemi di gestione delle basi di dati}
}

\newglossaryentry{message-broker}
{
    name=Message Broker,
    text=message broker,
    sort=Message Broker,
    description={È un componente che traduce un messaggio dalla rappresentazione formale di protocollo di un messaggio del mittente a quello del ricevente}
}

\newglossaryentry{dependency-injection}
{
    name=Dependency Injection,
    text=dependency injection,
    sort=Dependency Injection,
    description={\glsdisp{design-pattern}{Design pattern}\gloss\ architetturale della programmazione a oggetti in cui un oggetto fornisce le dipendenze a un altro oggetto. Una ``dipendenza'' può essere vista come un oggetto che può essere usato, come un servizio}
}
 % database di termini
\makeglossaries


%**************************************************************
% Impostazioni di graphicx
%**************************************************************
\graphicspath{{immagini/}} % cartella dove sono riposte le immagini


%**************************************************************
% Impostazioni di hyperref
%**************************************************************
\hypersetup{
    %hyperfootnotes=false,
    %pdfpagelabels,
    %draft,	% = elimina tutti i link (utile per stampe in bianco e nero)
    colorlinks=true,
	linktocpage=true,
    pdfstartpage=1,
    pdfstartview=FitV,
    % decommenta la riga seguente per avere link in nero (per esempio per la stampa in bianco e nero)
    %colorlinks=false, linktocpage=false, pdfborder={0 0 0}, pdfstartpage=1, pdfstartview=FitV,
    breaklinks=true,
    pdfpagemode=UseNone,
    pageanchor=true,
    pdfpagemode=UseOutlines,
    plainpages=false,
    bookmarksnumbered,
    bookmarksopen=true,
    bookmarksopenlevel=1,
    hypertexnames=true,
    pdfhighlight=/O,
    %nesting=true,
    %frenchlinks,
    urlcolor=webbrown,
    linkcolor=black,
    citecolor=webgreen,
    %pagecolor=RoyalBlue,
    %urlcolor=Black, linkcolor=Black, citecolor=Black, %pagecolor=Black,
    pdftitle={\myTitle},
    pdfauthor={\textcopyright\ \myName, \myUni, \myFaculty},
    pdfsubject={},
    pdfkeywords={},
    pdfcreator={pdfLaTeX},
    pdfproducer={LaTeX}
}

%**************************************************************
% Impostazioni di itemize
%**************************************************************
%\renewcommand{\labelitemi}{$\ast$}

\renewcommand{\labelitemi}{$\cdot$}
\renewcommand{\labelitemii}{$\circ$}
%\renewcommand{\labelitemiii}{$\diamond$}
%\renewcommand{\labelitemiv}{$\ast$}


%**************************************************************
% Impostazioni di listings
%**************************************************************
\lstset{
    language=Java,
    keywordstyle=\color{BurntOrange}\bfseries,
    basicstyle=\ttfamily,
%    identifierstyle=\color{Green},
    commentstyle=\color{Bittersweet}\ttfamily,
    stringstyle=\rmfamily\color{Green},
    numbers=none, %left,%
    numberstyle=\scriptsize, %\tiny
    stepnumber=5,
    numbersep=8pt,
    showstringspaces=false,
    breaklines=true,
    frameround=ftff,
    frame=none,
}


%**************************************************************
% Impostazioni di xcolor
%**************************************************************

\definecolor{webgreen}{rgb}{0,.5,0}
\definecolor{webbrown}{rgb}{.6,0,0}


%**************************************************************
% Altro
%**************************************************************

\newcommand{\omissis}{[\dots\negthinspace]} % produce [...]

% eccezioni all'algoritmo di sillabazione
\hyphenation
{
    ma-cro-istru-zio-ne
    gi-ral-din
}

\newcommand{\sectionname}{sezione}
\addto\captionsitalian{\renewcommand{\figurename}{Figura}
                       \renewcommand{\tablename}{Tabella}}

% \newcommand{\glsfirstoccur}{\ap{{[g]}}}
%\newcommand{\gloss}[1]{\textit{#1}\ped{\tiny{G}}}

\newcommand{\gloss}{\ped{\tiny{G}}}

\newcommand{\intro}[1]{\emph{\textsf{#1}}}

%**************************************************************
% Environment per ``rischi''
%**************************************************************
\newcounter{riskcounter}                % define a counter
\setcounter{riskcounter}{0}             % set the counter to some initial value

%%%% Parameters
% #1: Title
\newenvironment{risk}[1]{
    \refstepcounter{riskcounter}        % increment counter
    \par \noindent                      % start new paragraph
    \textbf{\arabic{riskcounter}. #1}   % display the title before the
                                        % content of the environment is displayed
}{
    \par\medskip
}

\newcommand{\riskname}{Rischio}

\newcommand{\riskdescription}[1]{\textbf{\\Descrizione:} #1.}

\newcommand{\risksolution}[1]{\textbf{\\Soluzione:} #1.}


%********************
% * Indici personalizzati *
%********************

% Indice per Code Snippets
\makeatletter
\newcommand\listcodesname{Elenco degli snippet}
\newcommand\listofcodes{%
	\chapter*{\listcodesname}\@starttoc{codes}}
\makeatother

% Indice per Use cases
\makeatletter
\newcommand\listucname{Elenco degli Use Case}
\newcommand\listofuc{%
	\chapter*{\listucname}\@starttoc{uc}}
\makeatother


% ****************************
% * Environment per ``use case''   *
% ****************************

% new tcolorbox environment for use cases
% #1: tcolorbox options
% #2: color
% #3: box title
\newtcolorbox{usecasebox}[3][]
{
	colframe = #2!25,
	colback  = #2!10,
	coltitle = #2!20!black,
	title    = {\bfseries #3},
	parbox=false,
}

\newcounter{usecasecounter}             % define a counter
\setcounter{usecasecounter}{0}          % set the counter to some initial value

%%%% Parameters
% #1: ID
% #2: Servizio
% #3: Nome
\newenvironment{usecase}[3]{
	\medskip

	\setlength{\parskip}{0pt}
	\newcommand{\usecasetitle}{\usecasename #1-#2: #3}
	\usecasebox{lightgray}{\large\textbf{\usecasetitle}}
	\renewcommand{\theusecasecounter}{\usecasename #1-#2}
 	\refstepcounter{usecasecounter}             % increment counter
	\addcontentsline{uc}{section}{\usecasetitle}
}{\endusecasebox%
	\medskip%
}

\newcommand{\usecasename}{UC}

\newcommand{\usecaseactors}[1]{\textbf{\\Attori Principali:} #1.}% \vspace{4pt}}
\newcommand{\usecasepre}[1]{\textbf{\\Precondizioni:} #1.}% \vspace{4pt}}
\newcommand{\usecasedesc}[1]{\textbf{\\Descrizione:} #1.}% \vspace{4pt}}
\newcommand{\usecasepost}[1]{\textbf{\\Postcondizioni:} #1.}% \vspace{4pt}}
\newcommand{\usecasealt}[1]{\textbf{\\Scenario Alternativo:} #1.}% \vspace{4pt}}

\newenvironment{ucitemize}%
{\begin{itemize}[topsep=0pt, partopsep=0pt]%
		\setlength{\itemsep}{2pt}%
		\setlength{\parskip}{0pt}}%
	{\end{itemize}}

\newenvironment{ucenumerate}%
{\enumerate[topsep=3pt, partopsep=0pt]%
		\setlength{\itemsep}{2pt}%
		\setlength{\parskip}{0pt}}%
	{\endenumerate}


\newenvironment{ucestensioni}[0]{%
	\vspace{4pt}%

	\textbf{Estensioni:}
	\ucenumerate%
}{\enducenumerate}

\newenvironment{ucgeneralizzazioni}{
	\vspace{4pt}%

	\textbf{Specializzazioni:}%
	\ucenumerate%
}{\enducenumerate}

\newenvironment{ucscenarioprincipale}{
	\vspace{4pt}
	
	\textbf{Scenario principale:}
	\ucenumerate
}{
	\enducenumerate
}

%**************************************************************
% Requisiti
%**************************************************************

\newcounter{reqcounter}
%\setcounter{reqcounter}{0}
\newcounter{subreqcounter}[reqcounter]

\newcommand{\reqrow}[3]{%
	\stepcounter{reqcounter}%
	#1-\thereqcounter & #2 & #3\\%
}

\newcommand{\reqtwocol}[2]{%
	\stepcounter{reqcounter}%
	#1-\thereqcounter & #2\\%
}

\newcommand{\subreqtwocol}[2]{%
	\stepcounter{subreqcounter}%
	#1-\thereqcounter.\thesubreqcounter & #2\\%
}


%**************************************************************
% Environment per ``namespace description''
%**************************************************************

\newenvironment{namespacedesc}{
    \vspace{10pt}
    \par \noindent                              % start new paragraph
    \begin{description}
}{
    \end{description}
    \medskip
}

\newcommand{\classdesc}[2]{\item[\textbf{#1:}] #2}


% subsubsection 4
% paragraph 5
% ecc..
\setcounter{secnumdepth}{4}
\setcounter{tocdepth}{4}


% Per tabularx con padding, parametro tra [], eg [1.3]
\newenvironment{paddedtablex}[1][1]{%
	\renewcommand*{\arraystretch}{#1}%
%	\renewcommand\theadfont{\bfseries}%
	\tabularx%
}{\endtabularx}
