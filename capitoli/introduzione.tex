% !TEX encoding = UTF-8
% !TEX TS-program = pdflatex
% !TEX root = ../tesi.tex

%**************************************************************
\chapter{Introduzione}
\label{cap:introduzione}
%**************************************************************

%Introduzione al contesto applicativo.\bigskip

\intro{Sarà mostrata nel presente capitolo l'azienda ospitante, Sync Lab, e la descrizione del progetto di stage.}

%\noindent Esempio di utilizzo di un termine nel glossario \par
%\noindent \gls{api}.\bigskip
%
%\noindent Esempio di citazione in linea \par
%\noindent \cite{site:agile-manifesto}. \bigskip
%
%\noindent Esempio di citazione nel pie' di pagina \par
%\noindent citazione\footcite{womak:lean-thinking}

%**************************************************************

\section{Convenzioni tipografiche}

% \begin{description}
% \item[{\hyperref[cap:processi-metodologie]{Il secondo capitolo}}] descrive ...

% \item[{\hyperref[cap:descrizione-stage]{Il terzo capitolo}}] approfondisce ...

% \item[{\hyperref[cap:analisi-requisiti]{Il quarto capitolo}}] approfondisce ...

% \item[{\hyperref[cap:progettazione-codifica]{Il quinto capitolo}}] approfondisce ...

% \item[{\hyperref[cap:verifica-validazione]{Il sesto capitolo}}] approfondisce ...

% \item[{\hyperref[cap:conclusioni]{Nel settimo capitolo}}] descrive ...
% \end{description}

Riguardo la stesura del testo, relativamente al documento sono state adottate le seguenti convenzioni tipografiche:
\begin{itemize}
	\item Gli acronimi, le abbreviazioni e i termini ambigui o di uso non comune menzionati vengono definiti nel glossario, situato alla fine del presente documento;
	\item Per la prima occorrenza dei termini riportati nel glossario viene utilizzata la seguente nomenclatura: \gloss{parola}.
	\item I termini in lingua straniera o facenti parti del gergo tecnico sono evidenziati con il carattere \emph{corsivo}.
\end{itemize}

%**************************************************************

\section{L'azienda}

\begin{figure}[H]
	\centering
	\includegraphics[width=\textwidth/2]{immagini/logo-synclab.png}
	\caption{Logo Sync Lab}
\end{figure}

\subsection{Profilo aziendale}
\myCompany\ è una società di consulenza informatica fondata nel 2002 con sedi a Napoli, Roma, Milano e Padova.

Fin dai primi anni Sync Lab è rapidamente cresciuta nel mercato ICT, consolidando i rapporti con clienti e partner, raggiungendo un organico aziendale di oltre 200 risorse,
una solida base finanziaria e una diffusione sul territorio attraverso le sue quattro sedi.

L'organico aziendale è andato crescendo in modo continuo e rapido, in relazione
all'apertura delle varie sedi ed alla progressiva crescita delle stesse.

La grande attenzione alla gestione delle risorse umane ha fatto di Sync Lab un
riferimento in positivo per quanti volessero avviare o far evolvere in chiave professionale la propria carriera.

Il basso turn-over testimonia la voglia dei collaboratori di condividere il progetto comune, assumendo all'interno di esso ruoli e responsabilità che solo un processo
evolutivo così intenso può offrire.
I ricavi hanno avuto un incremento proporzionale alla crescita dell'azienda beneficiando dell'approccio adattativo e diversificato al mercato.

\subsection{Servizi}

\myCompany\ è un'azienda leader nella consulenza tecnologica, impegnata in un
processo continuo di identificazione e messa in opera di soluzioni per i clienti finalizzate
alla creazione di valore.

L'azienda supporta le esigenze di innovazione di tutte le
organizzazioni ed in ogni settore di mercato nell'ambito \textit{Information Technology}, con
servizi in ambito:
\begin{itemize}
	\item \textit{Business Consultancy};
	\item \textit{Project Financing};
	\item \textit{IT Consultancy};
\end{itemize}

Sync Lab ha come punti di forza la qualità dei servizi offerti (certificazioni ISO 9001,
ISO 14001, ISO 27001, OHSAS 18001) ed un'accurata gestione delle risorse umane.
L'approfondita conoscenza di processi e tecnologie, maturata in esperienze altamente
significative e qualificanti, fornisce l'\textit{expertise} e il \textit{know-how} necessari per gestire
progetti di elevata complessità, dominando l'intero ciclo di vita: Studio di fattibilità,
Progettazione, Implementazione, \textit{Governance} e \textit{Post Delivery}.

L'offerta di consulenza specialistica trova le punte di eccellenza nella progettazione di
architetture software avanzate, siano esse per applicativi di dominio, per sistemi di
supporto al business (BSS), per sistemi di integrazione (EAI/SOA) o per sistemi di
monitoraggio applicativo/territoriale.

Il laboratorio RD (Ricerca e Sviluppo) dell'azienda è sempre al passo con i nuovi
paradigmi tecnologici e di comunicazione, ad esempio \textit{Big Data}, \textit{Cloud Computing},
\textit{Internet of Things}, \textit{Mobile e Sicurezza IT}, per supportare i propri clienti nella creazione
ed integrazione di applicazioni, processi e dispositivi.

Le attività in ambito \textit{Educational} ed RD hanno permesso di acquisire una profonda
conoscenza degli strumenti di finanza agevolata fruendone direttamente ed interagendo
con enti di supporto ai progetti innovativi dei propri clienti. L'azienda, grazie alla rete
di relazioni a livello nazionale ed internazionale, ha ottenuto importanti finanziamenti
in progetti RD europei (FP7 e H2020).

\subsection{Settori d'impiego}

Sync Lab si sta sempre più specializzando in vari settori d'impiego: dal mondo \textit{banking}
all'\textit{assurance} con una nicchia importante nell'ambito sanità in cui vanta un prodotto
d'eccellenza per la gestione delle cliniche private.
L'azienda inoltre ha recentemente fondato una collegata \textbf{Sync Security} che si occupa
espressamente del mondo della \textit{cyber security} e sicurezza informatica in genere.

%**************************************************************

\section{Lo stage}

Lo stage si è svolto nella sede di Padova dell'azienda ospitante, Sync Lab, nell'arco delle otto settimane previste nel piano di lavoro:
\begin{itemize}
	\item inizio il 6 maggio 2019;
	\item fine il 28 giugno 2019.
\end{itemize}
Il progetto di stage consiste nell'implementazione di una \textit{web application} a microservizi per il trasferimento in una nuova piattaforma del servizio di \textit{skill matrix} usato da Sync Lab per organizzare le competenze delle persone che vanno in azienda per un colloquio.

Prima di ogni colloquio a Sync Lab, è richiesto al candidato di compilare un foglio di calcolo contenente le competenze di interesse
per l'azienda e il livello, valore che il candidato dovrà indicare.

L'idea è quella di un portale web in cui il candidato compilerà i livelli delle competenze, in alternativa al vecchio foglio di calcolo.
Per fare ciò, è stato scelto un approccio moderno: 
\begin{itemize}
	\item il \textit{front-end} utilizzerà \textit{framework} di ultima generazione (nel nostro caso, Angular);
	\item il \textit{back-end} utilizzerà un'architettura a microservizi, i quali saranno implementati con il framework Spring (Java), per questioni di uniformità e semplicità, con MongoDB come DBMS.
\end{itemize}

A svolgere tale progetto, ci saranno altri stagisti oltre al sottoscritto, a cui a ognuno sarà assegnato un equo carico di lavoro.

Per quanto riguarda il back-end, l'azienda possiede già un portale web usato per caricare i dati dei candidati. Il DBMS utilizzato per i dati di tale portale è Oracle SQL.
Parte del lavoro dello stage consisterà nello spostare e riadattare il codice del vecchio portale a quello del nuovo.

Il mio ruolo nel progetto consisterà prettamente nell'implementazione di alcuni microservizi back-end e nel riadattamento del vecchio portale.
%**************************************************************
