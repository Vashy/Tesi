% !TEX encoding = UTF-8
% !TEX TS-program = pdflatex
% !TEX root = ../tesi.tex

%**************************************************************
\chapter{Descrizione dello stage}\label{cap:descrizione-stage}
%**************************************************************

\intro{Breve introduzione al capitolo}

%**************************************************************
\section{Introduzione al progetto}

%**************************************************************
\section{Analisi preventiva dei rischi}

Durante la fase di analisi iniziale sono stati individuati alcuni possibili rischi a cui si potrà andare incontro.
Si è quindi proceduto a elaborare delle possibili soluzioni per far fronte a tali rischi.

\begin{risk}{Performance del portatile utilizzato}
    \riskdescription{le performance del portatile, essendo ormai datato, potrebbero risultare lente o non abbastanza buone da causare dei rallentamenti sulle
        attività lavorative e di studio durante il periodo di stage}
    \risksolution{cercare di limitare gli sprechi di memoria utilizzando applicazioni non necessarie, e coinvolgere il tutor aziendale per l'eventuale sostituzione
        del PC}
    \label{risk:hardware-simulator}
\end{risk}

%**************************************************************
\section{Requisiti e obiettivi}


%**************************************************************
\section{Pianificazione}
