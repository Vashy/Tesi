% !TEX encoding = UTF-8
% !TEX TS-program = pdflatex
% !TEX root = ../tesi.tex

%**************************************************************
\chapter{Descrizione dello stage}\label{cap:descrizione-stage}

\intro{Breve introduzione al capitolo}

%**************************************************************

\section{Il progetto}

TODO

\section{Piano di lavoro}

Viene riportato in seguito i contenuti di maggior rilievo presenti nel piano di stage che ha pianificato
le attività dello stesso.

\subsection{Obiettivi}

\begin{itemize}[noitemsep]
	\item Obbligatori
	\begin{itemize}
		\item \underline{\textit{O01}}: Acquisizione competenze previste dal programma;
		\item \underline{\textit{O02}}: Capacità di raggiungere gli obiettivi richiesti in autonomia seguendo il crono-programma;
		\item \underline{\textit{O03}}: Portare a termine le modifiche richieste dal cliente con una percentuale di superamento pari al 50\%.
	\end{itemize}
	\item Desiderabili
	\begin{itemize}
		\item \underline{\textit{D01}}: Portare a termine le modifiche richieste dal cliente con una percentuale di superamento pari all'80\%.
	\end{itemize}
	\item Facoltativi
	\begin{itemize}
		\item \underline{\textit{F01}}: Acquisizione competenze sul framework Spring Cloud.
	\end{itemize}
\end{itemize}

%**************************************************************

%\section{Analisi preventiva dei rischi}
%
%Durante la fase di analisi iniziale sono stati individuati alcuni possibili rischi a cui si potrà andare incontro.
%Si è quindi proceduto a elaborare delle possibili soluzioni per far fronte a tali rischi.

\begin{risk}{Performance del portatile utilizzato}
    \riskdescription{le performance del portatile, essendo ormai datato, potrebbero risultare lente o non abbastanza buone da causare dei rallentamenti sulle
        attività lavorative e di studio durante il periodo di stage}
    \risksolution{cercare di limitare gli sprechi di memoria utilizzando applicazioni non necessarie, e coinvolgere il tutor aziendale per l'eventuale sostituzione
        del PC}
    \label{risk:hardware-simulator}
\end{risk}

%**************************************************************
\section{Requisiti e obiettivi}


%**************************************************************
\section{Pianificazione}
