% !TEX encoding = UTF-8
% !TEX TS-program = pdflatex
% !TEX root = ../tesi.tex

%**************************************************************
\chapter{Tecnologie e progettazione}
\label{cap:progettazione-codifica}
%**************************************************************

\intro{Nel presente capitolo vengono presentate gli strumenti utilizzati e la progettazione di base dei microservizi e degli endpoint.}

%**************************************************************
\section{Tecnologie e strumenti}
\label{sec:tecnologie-strumenti}

Di seguito viene fornita una panoramica delle tecnologie e strumenti utilizzati durante il periodo di stage.

%\subsection*{Spring}
%Descrizione Tecnologia 1.

\subsection{Ambiente di sviluppo}
Il percorso di stage è stato effettuato in due sistemi operativi differenti:
\begin{itemize}
	\item nella prima metà, ho utilizzato \textit{Linux Mint 19};
	\item nella seconda metà, ho fatto un passaggio a \textit{Ubuntu KDE 19}
\end{itemize}

Come \acrshort{ide}, ho variato provando un paio di alternative:
\begin{itemize}
	\item \textit{Spring Tool Suite 4}, un \textit{fork} di \textit{Eclipse} mantenuto da Pivotal, il team a cui appartiene \gls{spring};
	\item \textit{IntelliJ IDEA Ultimate Edition}, fornito gratuitamente grazie allo status di studente universitario.
\end{itemize} 

Entrambi gli \acrshort{ide} offrono un'ottima esperienza d'uso per linguaggio \gls{java}, con \gls{spring}.
Come personale preferenza, ho gradito molto di più \textit{IntelliJ IDEA}, strumento che ritengo superiore sia sotto il punto dell'interfaccia, molto più accattivante, che delle funzionalità, visto il potenziale incremento della produttività. Ciò è dovuto ai tool e vantaggi offerti da IntelliJ rispetto al concorrente, che ho trovato molto più carente sotto questo punto di vista.

\subsection{Strumenti di versionamento}

Per il versionamento del codice del progetto SyncRec, mi è stato fornito dal tutor aziendale \fabio\ accesso a una \gls{repository} privata, versionata tramite il \gls{vcs} \gls{git}.

\section{Linguaggi e framework}

Seguono la descrizione generale di alcuni linguaggi e \textit{framework} appresi durante il periodo di tirocinio.
La parte relativa a \glsdisp{microservizio}{microservizi} e \gls{spring-cloud}\gloss\ sono rispettivamente nelle appendici \S\ref{cap:microservizi} e \S\ref{cap:spring-cloud}, poiché è stato svolto un lavoro di analisi più approfondito.

\subsection{Java} Il principale linguaggio utilizzato per lo stage è stato \gls{java}. Questa scelta è stata dettata dall'azienda, poiché è il linguaggio che più utilizza per quanto riguarda il \textit{back-end}.

Viene ampiamente utilizzato dalle aziende italiane da molti anni, ed è per questo che tutt'ora è uno tra i linguaggi più richiesti, almeno nel panorama italiano.

Per le attività di stage ho cercato di utilizzare quanto più possibile le novità delle ultime versioni del \gls{jdk}, in particolare Java 8, quali ad esempio \texttt{Optional<T>}, le \textit{lamdba expressions}, \texttt{var}, gli \texttt{stream}, etc\dots

\subsection{Spring}

\gls{spring} è un \textit{framework} basato su linguaggio \gls{java} che permette lo sviluppo di applicazioni \textit{enterprise} praticamente per qualsiasi ambito.
È un \textit{framework} modulare, questo significa che non viene incluso \gls{spring} nella sua interezza: vanno inclusi solo i moduli di cui si necessita.

\subsubsection{Spring Core}

\textit{Core} è il modulo cuore di \gls{spring}, il fondamento del \textit{framework} stesso.

Il componente più importante di \textit{Spring Core} è l'\gls{ioc} container. Il ruolo principe di tale componente è quello di \gls{dependency-injection}\gloss.
Ogni classe gestita dal \gls{ioc} container dichiara le proprie dipendenze esplicitamente tramite costruttore (\textit{constructor injection}), tramite i \textit{setter} o tramite un \textit{factory method}.
In questo modo, il container si occupa di iniettare tali dipendenze quando crea il \textit{bean} (\S\ref{section:spring-beans})

\paragraph*{Spring Beans}\label{section:spring-beans} I \textit{bean} sono oggetti gestiti dall'\gls{ioc} container, il quale si occupa di costruirli e fornirli come dipendenze. Per la costruzione, il container utilizza i seguenti metadati legati al \textit{bean}:
\begin{itemize}
	\item \textit{class};
	\item \textit{name};
	\item \textit{scope};
	\item \textit{constructor arguments};
	\item \textit{properties};
	\item \textit{autowiring mode};
	\item \textit{lazy initialization mode};
	\item \textit{initialization method};
	\item \textit{destruction method};
\end{itemize}

Vediamo un esempio di \textit{constructor injection}:

\begin{tcolorbox}
	\begin{lstlisting}
@Component
public class SimpleClass {

  // the SimpleClass has a dependency on a
  // OtherClass
  private OtherClass otherObject;
	
  // a constructor so that the Spring container can
  // inject a OtherClass object
  public SimpleClass(OtherClass otherObject) {
    this.otherObject = otherObject;
  }

  // business logic that actually uses the
  // injected OtherClass is omitted...
}
	\end{lstlisting}
\end{tcolorbox}\addcontentsline{codes}{section}{Esempio di Constructor Injection}

La dipendenza \texttt{OtherClass} viene iniettata dall'\gls{ioc} container in fase di inizializzazione del componente, se è presente un \textit{bean} di tipo \texttt{OtherClass}.
Nell'eventualità in cui siano presenti più di un costruttore, è necessario annotare con \texttt{@Autowired} il costruttore che dev'essere preso in considerazione dal container.

Ci sono sei tipi di \textit{scoping} dei \textit{bean}:
\begin{itemize}
	\item \textit{singleton}: è presente singola istanza per ogni \gls{ioc} container;
	\item \textit{prototype}: possono essere presenti più istanze per ogni container;
	\item \textit{request}: ogni istanza è mappata sul ciclo di vita di una singola richiesta \acrshort{http};
	\item \textit{session}: ogni istanza è mappata sul ciclo di vita di una sessione \acrshort{http};
	\item \textit{application}: ogni istanza è mappata sul ciclo di vita di un oggetto \texttt{ServletContext};
	\item \textit{websocket}: ogni istanza è mappata sul ciclo di vita di un oggetto \texttt{WebSocket}.
\end{itemize}

Lo \textit{scope} di default dei \textit{bean} è \textit{singleton}. Ogni \textit{bean} viene istanziato una sola volta e viene pertanto fornita sempre la stessa istanza agli oggetti che richiedono tale dipendenza.

\subsubsection{Altri moduli di Spring}

\paragraph*{Spring Boot} Modulo di \gls{spring} che permette l'avvio di applicazioni in modo molto più automatico rispetto ad applicazioni non \textit{Boot}. Parte con molte configurazioni di default che altrimenti necessiterebbero di configurazione manuali.

\paragraph*{Spring Data MongoDB} Modulo che permette una gestione ad alto livello del \acrshort{dbms} \gls{mongodb}.
Per effettuare operazioni sul database è sufficiente dichiarare un'interfaccia che estenda \texttt{MongoRepository}, senza il bisogno di conoscere il relativo \gls{dql}.

\paragraph*{Spring Data REST} Permette di sviluppare servizi RESTful semplicemente dichiarando la classe di modello.
Gli \textit{endpoint} di visualizzazione, aggiunta, modifica e rimozione sono definiti automaticamente dal \textit{framework}.

\subsection{Angular}

\gls{angular} è un \textit{framework} scritto in \gls{typescript} per lo sviluppo di applicazioni web.
Permette di estendere \acrshort{html} con attributi chiamati direttive.

\begin{tcolorbox}
	\begin{lstlisting}[language=html]
<h2>Products</h2>
<div *ngFor="let product of products">
    <h3>{{ product.name }}</h3>
</div>
	\end{lstlisting}
\end{tcolorbox}\addcontentsline{codes}{section}{Esempio della parte HTML di un component Angular}

In questo esempio, è possibile vedere la direttiva \texttt{ngFor}. Per ogni \texttt{product} nella lista \texttt{products}, produce un elemento \texttt{h3} contenente il nome del prodotto, sostituendo il \texttt{div}. Tale lista è definita nel file contenente il codice \gls{typescript} relativo al \textit{component}.

\paragraph*{Modules}
Le applicazioni \gls{angular} sono modulari e per tale fine ci sono gli \texttt{NgModules}: sono \textit{container} per un blocco di codice coeso dedicato a un dominio.

I moduli possono contenere più \textit{component}, \textit{service provider} e altri file di codice generici. Ogni \texttt{NgModule} è definito da una classe con il \textit{decorator} \texttt{@NgModule()}. Essa prende come argomento un oggetto che può definire molte proprietà, tra cui:
\begin{itemize}
	\item \textit{declarations};
	\item \textit{exports};
	\item \textit{imports};
	\item \textit{providers};
	\item \textit{bootstrap}.
\end{itemize}

\paragraph*{Components}
Definiscono le viste. Sono un insieme di elementi a schermo tra cui \gls{angular} può scegliere e modificare in base alla logica dell'applicazione.
Ogni componente viene inizializzato con una classe, un template \acrshort{html} e uno stile specifico (\acrshort{css}).
Per marcare una classe come \textit{component}, va usato il \textit{decorator} \texttt{@Component}, in cui è possibile definire alcune proprietà, tra cui:
\begin{itemize}
	\item \textit{selector};
	\item \textit{templateUrl};
	\item \textit{providers}.
\end{itemize}

\paragraph*{Services}
Forniscono funzionalità specifiche non correlate direttamente con le viste. I \textit{service} possono essere iniettati nei \textit{component} come dipendenze tramite \gls{dependency-injection}.

%**************************************************************
%\section{Ciclo di vita del software}
%\label{sec:ciclo-vita-software}

%**************************************************************
\section{Progettazione}\label{sec:progettazione}

Sono spiegati nella presente sezione i dettagli fondamentali riguardanti l'architettura dei microservizi e gli \textit{endpoint} usati per il progetto SyncRec.

\subsection{Architettura}
Ogni \gls{microservizio} conterrà i \textit{package} che seguono:

\begin{itemize}
	\item \texttt{controller}: contiene tutti i \textit{controller} personalizzati;
	\item \texttt{model}: contiene le classi modello che saranno usate per interagire con il \textit{database};
	\item \texttt{repository}: contiene le dichiarazioni delle \textit{repository} di \gls{spring};
	\item \texttt{service}: contiene i \textit{service} di \gls{spring};
	\item \texttt{config}: contiene le configurazioni del servizio;
	\item \texttt{validation}: contiene le classi usate per validare i dati prima di essere inseriti tramite le \textit{repository} nel \textit{database};
	\item \texttt{converter}: package contenente i convertitori definiti dallo sviluppatore.
\end{itemize}

Inoltre, allo stesso livello di questi \textit{package} sarà presente la classe \texttt{Application.java}, contenente il \textit{main}
che avvia la \texttt{SpringApplication}.

Ogni servizio potrà definire \textit{package} personalizzati per gestire logiche e modelli non riconducibili a \gls{spring}.


\subsection{Progettazione dei microservizi}\label{prog-microservizi}

\subsubsection{Microservizio: Email Service}

Il \gls{microservizio} email ha il semplice compito di inviare una email specifica.
Testo, oggetto e destinatario sono dati definiti nel database \gls{mongodb}
\textit{syncrec-email}, mentre le proprietà del server \acrshort{smtp} sono definite in un file di configurazione.

L'invio della email avviene in modo sincrono quando viene effettuata una richiesta POST tramite l'endpoint \texttt{/sendEmail}.
I campi da immettere nel corpo della richiesta sono:
\begin{itemize}
	\item \texttt{receiver}: il destinatario della mail.
\end{itemize}

La seguente potrebbe essere un esempio di richiesta al servizio Email:

\begin{tcolorbox}
	\begin{verbatim} curl -X POST http://localhost:8080/syncrec/sendEmail -d \
	'{ "receiver":  "emailReceiver@email.it" }'
	\end{verbatim}
\end{tcolorbox}\addcontentsline{codes}{section}{Esempio richiesta POST per microservizio email}

Come risposta, è possibile ottenere 2 risultati:
\begin{itemize}
	\item \textbf{Ok} (200) e un \textit{payload} contenente i campi usati
	per l'invio email;
	\item \textbf{Unprocessable entity} (406) se non è stato possibile inviare l'email (per errori di connessioni al database, al server \acrshort{smtp} etc\dots).
\end{itemize}

\clearpage
Con ``proprietà'' ci si riferirà ai campi usati per l'invio email, ossia ``sender'', ``password'' o ``subject''.

\begin{table}[H]
	\begin{paddedtablex}[1.7]{\textwidth}{cX}
		\textbf{Richiesta} & \textbf{Azione} \\\toprule
		\texttt{GET /props} & Visualizzazione delle proprietà salvate nel servizio\\
		\texttt{GET /props/\{id\}} & Visualizzazione della proprietà corrispondente a \texttt{id}\\
		\texttt{POST /props} & Aggiunta di una nuova proprietà al servizio\\
		\texttt{PUT /props/\{id\}} & Sostituzione della proprietà corrispondente a \texttt{id}\\
		\texttt{PATCH /props/\{id\}} & Modifica della proprietà corrispondente a \texttt{id}\\
		\texttt{DELETE /props/\{id\}} & Rimozione della proprietà corrispondente a \texttt{id}\\
		\texttt{POST /sendEmail} & Invio dell'email\\
		\bottomrule
	\end{paddedtablex}
	\caption{Endpoint del servizio Email Sender}
	\label{tab:endpoint-es}
\end{table}


\subsubsection{Microservizio: Login}

Questo \gls{microservizio} espone la risorsa \texttt{User} all'endpoint
\texttt{syncrec/users}, accessibile e modificabile via \acrshort{api} \acrshort{rest}\gloss. I dati persistenti sono salvati e acceduti tramite il database \textit{syncrec-users} di
\gls{mongodb}.

Inoltre, tramite l'endpoint \texttt{/login} e una richiesta POST,
è possibile effettuare l'operazione di login.

I parametri richiesti nella richiesta di login sono:
\begin{itemize}
	\item \texttt{username}: campo contenente lo \textit{username} o l'email dell'utente che vuole autenticarsi;
	\item \texttt{password}: campo contenente la password relativa all'utente da autenticare.
\end{itemize}

Le possibili risposte che verranno inoltrate al chiamante possono essere:
\begin{itemize}
	\item \textbf{Ok} (200) e il \acrshort{json}\gloss\ contenente i campi relativi all'utente autenticato;
	\item \textbf{Bad request} (400) se il corpo della richiesta non contiene i campi riportati sopra, con corpo contenente un messaggio d'errore (sempre in formato \acrshort{json});
	\item \textbf{Unauthorized} (401) se l'autenticazione fallisce, con corpo vuoto.
\end{itemize}

\begin{table}[H]
	\begin{paddedtablex}[1.7]{\textwidth}{cX}
		\textbf{Richiesta} & \textbf{Azione} \\\toprule
		\texttt{GET /users} & Visualizzazione degli utenti salvati nel servizio\\
		\texttt{GET /users/\{id\}} & Visualizzazione dell'utente corrispondente a \texttt{id}\\
		\texttt{POST /users} & Aggiunta di un nuovo utente al servizio\\
		\texttt{PUT /users/\{id\}} & Sostituzione dell'utente corrispondente a \texttt{id}\\
		\texttt{PATCH /users/\{id\}} & Modifica dell'utente corrispondente a \texttt{id}\\
		\texttt{DELETE /users/\{id\}} & Rimozione dell'utente corrispondente a \texttt{id}\\
		\texttt{POST /login} & Utility di login del servizio\\
		\bottomrule
	\end{paddedtablex}
	\caption{Endpoint del servizio Login}
	\label{tab:endpoint-l}
\end{table}

\subsubsection{Microservizio: Catalog}

È stato richiesto di implementare un \gls{microservizio} con il fine di restituire,
attraverso una richiesta \acrshort{http}, una lista di \textit{skill} di interesse dell'azienda.

Questo era precedentemente effettuato tramite un foglio di calcolo inviato preventivamente ai candidati, i quali erano tenuti a compilarlo.

\begin{table}[H]
	\begin{paddedtablex}[1.7]{\textwidth}{cX}
		\textbf{Richiesta} & \textbf{Azione} \\\toprule
		\texttt{GET /skills} & Visualizzazione delle \textit{skill} salvate nel servizio\\
		\texttt{GET /skills/search/byCategory} & Visualizzazione delle \textit{skill} appartenenti alla categoria specificata\\
		\texttt{GET /skills/\{id\}} & Visualizzazione della \textit{skill} corrispondente a \texttt{id}\\
		\texttt{POST /skills} & Aggiunta di una nuova \textit{skill} al servizio\\
		\texttt{PUT /skills/\{id\}} & Sostituzione della \textit{skill} corrispondente a \texttt{id}\\
		\texttt{PATCH /skills/\{id\}} & Modifica della \textit{skill} corrispondente a \texttt{id}\\
		\texttt{DELETE /skills/\{id\}} & Rimozione della \textit{skill} corrispondente a \texttt{id}\\
		\bottomrule
	\end{paddedtablex}
	\caption{Endpoint del servizio Catalog}
	\label{tab:endpoint-c}
\end{table}


\subsubsection{Microservizio: Applicant}

L'ultima maschera da implementare è il \gls{microservizio} di gestione dei candidati (\textit{applicants}).

Sarà possibile aggiungere, rimuovere o modificare i candidati tramite
le opportune chiamate \acrshort{rest} all'\textit{endpoint} \texttt{syncrec/applicants}, riportate nella tabella che segue.

% TODO Scommentare se c'è tempo per aggiungere i relativi casi d'uso
%Ogni candidato avrà associati dei documenti, i quali non è ragionevole passare tramite \textit{payload} \acrshort{json} in un servizio \acrshort{rest}, con il tipico \textit{header} di \acrshort{rest}: \texttt{Content-Type: application/json}.
%
%Per fare ciò, i documenti saranno salvati in un'apposita \textit{collection} (analogo di \gls{mongodb} alle tabelle di un database relazionale):
%\textit{documents}.
%Ogni documento avrà associato ID, il nome e il contenuto binario del file.
%L'aggiunta, la rimozione e la modifica di tali documenti avverrà tramite richieste con \textit{header} \texttt{Content-Type: multipart/form-data}
%all'\textit{endpoint} \texttt{syncrec/documents}.
%
%Nelle risorse \textit{applicants}, verrà salvato l'ID dei documenti associati, con un opportuno check in fase di salvataggio dell'entità.

\begin{table}[H]
	\begin{paddedtablex}[1.7]{\textwidth}{XX}
		\textbf{Richiesta} & \textbf{Azione} \\\toprule
		\texttt{GET /applicants} & Visualizzazione degli \textit{applicant} salvati nel servizio\\
		\texttt{GET /applicants/\{id\}} & Visualizzazione dell'\textit{applicant} corrispondente a \texttt{id}\\
		\texttt{GET /applicants/search/byEmail} & Visualizzazione degli \textit{applicant} che hanno l'email specificata\\
		\texttt{GET /applicants/search/bySeniority} & Visualizzazione degli \textit{applicant} che hanno la \textit{seniority} specificata\\
		\texttt{GET /applicants/search/byPhone} & Visualizzazione degli \textit{applicant} che hanno il numero di telefono specificato\\
		\texttt{GET /applicants/search/considerable} & Visualizzazione degli \textit{applicant} marcati come considerabili\\
		\texttt{POST /applicants} & Aggiunta di un nuovo \textit{applicant} al servizio\\
		\texttt{PUT /applicants/\{id\}} & Sostituzione dell'\textit{applicant} corrispondente a \texttt{id}\\
		\texttt{PATCH /applicants/\{id\}} & Modifica dell'\textit{applicant} corrispondente a \texttt{id}\\
		\texttt{DELETE /applicants/\{id\}} & Rimozione dell'\textit{applicant} corrispondente a \texttt{id}\\
		\bottomrule
	\end{paddedtablex}
	\caption{Endpoint del servizio Applicant}
	\label{tab:endpoint-a}
\end{table}
%
%\subsubsection{Namespace 1} %**************************
%Descrizione namespace 1.
%
%\begin{namespacedesc}
%    \classdesc{Classe 1}{Descrizione classe 1}
%    \classdesc{Classe 2}{Descrizione classe 2}
%\end{namespacedesc}
%
%
%%**************************************************************
%\section{Design Pattern utilizzati}
%
%%**************************************************************
%\section{Codifica}
