% !TEX encoding = UTF-8
% !TEX TS-program = pdflatex
% !TEX root = ../tesi.tex

%**************************************************************
\chapter{Progettazione e codifica}
\label{cap:progettazione-codifica}
%**************************************************************

\intro{Breve introduzione al capitolo}\\ %TODO

%**************************************************************
\section{Tecnologie e strumenti}
\label{sec:tecnologie-strumenti}

Di seguito viene fornita una panoramica delle tecnologie e strumenti utilizzati durante il periodo di stage.

%\subsection*{Spring}
%Descrizione Tecnologia 1.

\subsection{Ambiente di sviluppo}
Il percorso di stage è stato effettuato in due sistemi operativi differenti:
\begin{itemize}
	\item nella prima metà, ho utilizzato \textit{Linux Mint 19};
	\item nella seconda metà, ho fatto un passaggio a \textit{Ubuntu KDE 19}
\end{itemize}

Come \acrshort{ide}, ho variato provando un paio di alternative:
\begin{itemize}
	\item \textit{Spring Tool Suite 4}, un \textit{fork} di \textit{Eclipse} mantenuto da Pivotal, il team a cui appartiene \gls{spring};
	\item \textit{IntelliJ IDEA Ultimate Edition}, fornito gratuitamente grazie allo status di studente universitario.
\end{itemize} 

Entrambi gli \acrshort{ide} offrono un'ottima esperienza d'uso per linguaggio \gls{java}, con \gls{spring}.
Come personale preferenza, ho gradito molto di più \textit{IntelliJ IDEA}, strumento che ritengo superiore sia sotto il punto dell'interfaccia, molto più accattivante, che delle funzionalità, che potenzialmente incremento la produttività. Ciò è dovuto ai tool e vantaggi offerti da IntelliJ rispetto al concorrente, che ho trovato molto più carente sotto questo punto di vista.

\subsection{Strumenti di versionamento}

Per il versionamento del codice del progetto SyncRec, mi è stato fornito dal tutor aziendale \fabio\ accesso a una \gls{repository} privata, versionata tramite il \gls{vcs} \gls{git}.

\subsection{Linguaggi e framework}

\subsubsection{Java} Il principale linguaggio utilizzato per lo stage è stato \gls{java}. Questa scelta è stata dettata dall'azienda, poiché è il linguaggio che più utilizza per quanto riguarda il \textit{back-end}.

Viene ampiamente utilizzato dalle aziende italiane da molti anni, ed è per questo che tutt'ora è uno tra i linguaggi più richiesti, almeno nel panorama italiano.

Per le attività di stage ho cercato di utilizzare quanto più possibile le novità delle ultime versioni del \gls{jdk}, in particolare Java 8, quali ad esempio \texttt{Optional<T>}, le \textit{lamdba expressions}, \texttt{var}, gli \texttt{stream}, etc\dots

\subsubsection{Spring}

\gls{spring} è un \textit{framework} basato su linguaggio \gls{java} che permette lo sviluppo di applicazione \textit{enterprise} praticamente per qualsiasi ambito.
È un \textit{framework} modulare, questo significa che non viene incluso \gls{spring} nella sua interezza: vanno inclusi solo i moduli che servono per le attività.

\paragraph*{Spring Boot} Modulo di \gls{spring} che permette l'avvio di applicazioni in modo molto più automatico rispetto ad applicazioni non \textit{Boot}. Parte con molte configurazioni di default che altrimenti necessiterebbero di configurazioni manuali.

\paragraph*{Spring Data MongoDB} Modulo che permette una gestione ad alto livello del \acrshort{dbms} \gls{mongodb}.
Per effettuare operazioni sul database è sufficiente dichiarare un'interfaccia che estenda \texttt{MongoRepository}, senza il bisogno di conoscere il relativo \gls{dql}.

\paragraph*{Spring Data REST} Permette di sviluppare servizi RESTful semplicemente dichiarando la classe di modello.
Gli \textit{endpoint} di visualizzazione, aggiunta, modifica e rimozione sono definiti automaticamente dal \textit{framework}.

%**************************************************************
\section{Ciclo di vita del software}
\label{sec:ciclo-vita-software}

%**************************************************************
\section{Progettazione}\label{sec:progettazione}

Sono spiegati nella presente sezione i dettagli fondamentali riguardanti l'architettura dei microservizi e gli \textit{endpoint} usati per il progetto SyncRec.

\subsection{Architettura}
Ogni \gls{microservizio} conterrà i \textit{package} che seguono:

\begin{itemize}
	\item \texttt{controller}: contiene tutti i \textit{controller} personalizzati;
	\item \texttt{model}: contiene le classi modello che saranno usate per interagire con il \textit{database};
	\item \texttt{repository}: contiene le dichiarazioni delle \textit{repository} di \gls{spring};
	\item \texttt{service}: contiene i \textit{service} di \gls{spring};
	\item \texttt{config}: contiene le configurazioni del servizio;
	\item \texttt{validation}: contiene le classi usate per validare i dati prima di essere inseriti tramite le \textit{repository} nel \textit{database};
	\item \texttt{converter}: package contenente i convertitori definiti dallo sviluppatore.
\end{itemize}

Inoltre, allo stesso livello di questi \textit{package} sarà presente la classe \texttt{Application.java}, contenente il \textit{main}
che avvia la \texttt{SpringApplication}.

Ogni servizio potrà definire \textit{package} personalizzati per gestire logiche e modelli non riconducibili a \gls{spring}.

\subsection{Endpoint}

\subsubsection{EmailService}
Con ``proprietà'' ci si riferirà ai campi usati per l'invio email, ossia ``sender'', ``password'' o ``subject''.

\begin{table}[H]
	\begin{paddedtablex}[1.7]{\textwidth}{cX}
		\textbf{Richiesta} & \textbf{Azione} \\\toprule
		\texttt{GET /props} & Visualizzazione delle proprietà salvate nel servizio\\
		\texttt{GET /props/\{id\}} & Visualizzazione della proprietà corrispondente a \texttt{id}\\
		\texttt{POST /props} & Aggiunta di una nuova proprietà al servizio\\
		\texttt{PUT /props/\{id\}} & Sostituzione della proprietà corrispondente a \texttt{id}\\
		\texttt{PATCH /props/\{id\}} & Modifica della proprietà corrispondente a \texttt{id}\\
		\texttt{DELETE /props/\{id\}} & Rimozione della proprietà corrispondente a \texttt{id}\\
		\texttt{POST /sendEmail} & Invio dell'email\\
		\bottomrule
	\end{paddedtablex}
	\caption{Endpoint del servizio Email Sender}
	\label{tab:endpoint-es}
\end{table}


\subsubsection{Login}

\begin{table}[H]
	\begin{paddedtablex}[1.7]{\textwidth}{cX}
		\textbf{Richiesta} & \textbf{Azione} \\\toprule
		\texttt{GET /users} & Visualizzazione degli utenti salvati nel servizio\\
		\texttt{GET /users/\{id\}} & Visualizzazione dell'utente corrispondente a \texttt{id}\\
		\texttt{POST /users} & Aggiunta di un nuovo utente al servizio\\
		\texttt{PUT /users/\{id\}} & Sostituzione dell'utente corrispondente a \texttt{id}\\
		\texttt{PATCH /users/\{id\}} & Modifica dell'utente corrispondente a \texttt{id}\\
		\texttt{DELETE /users/\{id\}} & Rimozione dell'utente corrispondente a \texttt{id}\\
		\texttt{POST /login} & Utility di login del servizio\\
		\bottomrule
	\end{paddedtablex}
	\caption{Endpoint del servizio Login}
	\label{tab:endpoint-l}
\end{table}


\subsubsection{Catalog}

\begin{table}[H]
	\begin{paddedtablex}[1.7]{\textwidth}{cX}
		\textbf{Richiesta} & \textbf{Azione} \\\toprule
		\texttt{GET /skills} & Visualizzazione delle \textit{skill} salvate nel servizio\\
		\texttt{GET /skills/\{id\}} & Visualizzazione della \textit{skill} corrispondente a \texttt{id}\\
		\texttt{POST /skills} & Aggiunta di una nuova \textit{skill} al servizio\\
		\texttt{PUT /skills/\{id\}} & Sostituzione della \textit{skill} corrispondente a \texttt{id}\\
		\texttt{PATCH /skills/\{id\}} & Modifica della \textit{skill} corrispondente a \texttt{id}\\
		\texttt{DELETE /skills/\{id\}} & Rimozione della \textit{skill} corrispondente a \texttt{id}\\
		\bottomrule
	\end{paddedtablex}
	\caption{Endpoint del servizio Login}
	\label{tab:endpoint-c}
\end{table}


\subsubsection{Applicant}

\begin{table}[H]
	\begin{paddedtablex}[1.7]{\textwidth}{XX}
		\textbf{Richiesta} & \textbf{Azione} \\\toprule
		\texttt{GET /applicants} & Visualizzazione degli \textit{applicant} salvati nel servizio\\
		\texttt{GET /applicants/\{id\}} & Visualizzazione dell'\textit{applicant} corrispondente a \texttt{id}\\
		\texttt{GET /applicants/search/byEmail} & Visualizzazione degli \textit{applicant} che hanno l'email specificata\\
		\texttt{GET /applicants/search/bySeniority} & Visualizzazione degli \textit{applicant} che hanno la \textit{seniority} specificata\\
		\texttt{GET /applicants/search/byPhone} & Visualizzazione degli \textit{applicant} che hanno il numero di telefono specificato\\
		\texttt{GET /applicants/search/considerable} & Visualizzazione degli \textit{applicant} marcati come considerabili\\
		\texttt{POST /applicants} & Aggiunta di un nuovo \textit{applicant} al servizio\\
		\texttt{PUT /applicants/\{id\}} & Sostituzione dell'\textit{applicant} corrispondente a \texttt{id}\\
		\texttt{PATCH /applicants/\{id\}} & Modifica dell'\textit{applicant} corrispondente a \texttt{id}\\
		\texttt{DELETE /applicants/\{id\}} & Rimozione dell'\textit{applicant} corrispondente a \texttt{id}\\
		\bottomrule
	\end{paddedtablex}
	\caption{Endpoint del servizio Login}
	\label{tab:endpoint-a}
\end{table}


\subsubsection{Namespace 1} %**************************
Descrizione namespace 1.

\begin{namespacedesc}
    \classdesc{Classe 1}{Descrizione classe 1}
    \classdesc{Classe 2}{Descrizione classe 2}
\end{namespacedesc}


%**************************************************************
\section{Design Pattern utilizzati}

%**************************************************************
\section{Codifica}
