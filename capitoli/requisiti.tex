% !TEX encoding = UTF-8
% !TEX TS-program = pdflatex
% !TEX root = ../tesi.tex

%**************************************************************
\chapter{Analisi dei requisiti}
\label{cap:analisi-requisiti}
%**************************************************************

\intro{Breve introduzione al capitolo}\\ % TODO

\section{Casi d'uso}

Per lo studio dei casi di utilizzo del prodotto sono stati creati dei diagrammi.
I diagrammi dei casi d'uso (in inglese \emph{Use Case Diagram}) sono diagrammi di tipo \gls{uml} dedicati alla descrizione delle funzioni o servizi offerti da un sistema, così come sono percepiti e utilizzati dagli attori che interagiscono col sistema stesso.

\subsection{Nomenclatura del codice identificativo}

%\subparagraph{Denominazione }
Ogni codice presenta una forma del tipo:

\begin{center}
	\texttt{UC[Numero]-[Servizio]}
\end{center}

\begin{itemize}
	\item \textbf{UC}: sta per \textit{use case}, l'equivalente inglese di ``caso d'uso''.
	\item \textbf{Numero}: numero progressivo che si riferisce al caso. Se il caso d'uso è principale, è un semplice intero (e.g. 1 per il primo caso d'uso). Mentre se è un sotto caso, presenta due interi separati da un punto (e.g 1.1 per per il primo figlio del primo caso d'uso);
	\item \textbf{Servizio}: indica il servizio preso come riferimento per il caso d'uso.
	\begin{itemize}
		\item \textbf{ES}: Email Sender.
		\item \textbf{L}: User/Login.
		\item \textbf{C}: Catalog.
		\item \textbf{A}: Applicant.
	\end{itemize}
\end{itemize}

\subsection{Casi d'uso relativi al microservizio Email Sender}

\begin{usecase}{1}{ES}{Aggiunta proprietà per invio mail}
	\usecaseactors{sviluppatore SyncRec}

	\usecasepre{una nuova proprietà va aggiunta al servizio Email Sender}

	\usecasedesc{lo sviluppatore vuole aggiungere una nuova proprietà nel servizio}

	\begin{ucscenarioprincipale}
		\item L'utente inserisce la chiave da inserire;
		\item L'utente inserisce il valore della proprietà.
	\end{ucscenarioprincipale}

	\usecasepost{una nuova proprietà è stata aggiunta con successo al servizio}

	\begin{ucestensioni}
		\item \ref{uc:vis-errore-ins-proprieta-es}
	\end{ucestensioni}

	\begin{ucgeneralizzazioni}
		\item \ref{uc:aggiunta-sender-es}
		\item \ref{uc:aggiunta-password-es}
		\item \ref{uc:aggiunta-subject-es}
	\end{ucgeneralizzazioni}

	\label{uc:aggiunta-proprieta-es}
\end{usecase}

\begin{usecase}{2}{ES}{Visualizzazione errore inserimento proprietà}
	\usecaseactors{sviluppatore SyncRec}

	\usecasepre{c'è un tentativo di aggiunta di una proprietà}

	\usecasedesc{lo sviluppatore viene avvisato che c'è stato un tentativo di inserimento non valido: chiave non riconosciuta oppure valore non valido}

	\begin{ucscenarioprincipale}
		\item Lo sviluppatore visualizza il messaggio d'errore.
	\end{ucscenarioprincipale}

	\usecasepost{il messaggio d'errore è stato visualizzato}

	\label{uc:vis-errore-ins-proprieta-es}
\end{usecase}

\begin{usecase}{3}{ES}{Aggiunta proprietà ``sender''}
	\usecaseactors{sviluppatore SyncRec}

	\usecasepre{la proprietà ``sender'' va aggiunta al servizio Email Sender}

	\usecasedesc{lo sviluppatore vuole aggiungere la proprietà sender nel servizio Email Sender}

	\begin{ucscenarioprincipale}
		\item Lo sviluppatore seleziona la chiave ``sender'';
		\item Lo sviluppatore inserisce il valore della proprietà ``sender''.
	\end{ucscenarioprincipale}

	\usecasepost{la proprietà ``sender'' è stata aggiunta con successo}

	\label{uc:aggiunta-sender-es}
\end{usecase}

\begin{usecase}{4}{ES}{Aggiunta proprietà ``password''}
	\usecaseactors{sviluppatore SyncRec}

	\usecasepre{la proprietà ``password'' va aggiunta al servizio Email Sender}

	\usecasedesc{lo sviluppatore vuole aggiungere la proprietà ``password'' relativa al ``sender'' nel servizio Email Sender. Il servizio salverà la password criptandola con un opportuno algoritmo}

	\begin{ucscenarioprincipale}
		\item Lo sviluppatore seleziona la chiave ``password'';
		\item Lo sviluppatore inserisce il valore della proprietà ``password''.
	\end{ucscenarioprincipale}

	\usecasepost{la proprietà ``password'' è stata aggiunta con successo}
	\label{uc:aggiunta-password-es}
\end{usecase}

\begin{usecase}{5}{ES}{Aggiunta proprietà ``subject''}
	\usecaseactors{sviluppatore SyncRec}

	\usecasepre{la proprietà ``subject'' va aggiunta al servizio Email Sender}

	\usecasedesc{lo sviluppatore vuole aggiungere la proprietà ``subject'', l'oggetto della email da inviare, nel servizio Email Sender}

	\begin{ucscenarioprincipale}
		\item Lo sviluppatore seleziona la chiave ``subject'';
		\item Lo sviluppatore inserisce il valore della proprietà ``subject''.
	\end{ucscenarioprincipale}

	\usecasepost{la proprietà ``subject'' è stata aggiunta con successo}

	\label{uc:aggiunta-subject-es}
\end{usecase}


\begin{usecase}{6}{ES}{Modifica proprietà per invio mail}
	\usecaseactors{sviluppatore SyncRec}

	\usecasepre{una proprietà necessita di essere modificata}

	\usecasedesc{lo sviluppatore vuole modificare il valore di una nuova proprietà nel servizio}

	\begin{ucscenarioprincipale}
		\item Lo sviluppatore inserisce la chiave da modificare;
		\item Lo sviluppatore inserisce il nuovo valore della proprietà.
	\end{ucscenarioprincipale}

	\usecasepost{una proprietà è stata modificata con successo}

	\begin{ucestensioni}
		\item \ref{uc:vis-errore-mod-proprieta-es}
	\end{ucestensioni}

	\begin{ucgeneralizzazioni}
		\item \ref{uc:modifica-sender-es}
		\item \ref{uc:modifica-password-es}
		\item \ref{uc:modifica-subject-es}
	\end{ucgeneralizzazioni}

	\label{uc:modifica-proprieta-es}
\end{usecase}

\begin{usecase}{7}{ES}{Visualizzazione errore modifica proprietà}
	\usecaseactors{sviluppatore SyncRec}

	\usecasepre{c'è un tentativo di modifica di una proprietà}

	\usecasedesc{lo sviluppatore viene avvisato che c'è stato un tentativo di modifica non valido: chiave non riconosciuta oppure valore non valido}

	\begin{ucscenarioprincipale}
		\item Lo sviluppatore visualizza il messaggio d'errore.
	\end{ucscenarioprincipale}

	\usecasepost{il messaggio d'errore è stato visualizzato}

	\label{uc:vis-errore-mod-proprieta-es}
\end{usecase}

\begin{usecase}{8}{ES}{Modifica proprietà ``sender''}
	\usecaseactors{sviluppatore SyncRec}

	\usecasepre{la proprietà ``sender'' necessita di essere modificata}

	\usecasedesc{lo sviluppatore vuole modificare la proprietà ``sender'' nel servizio Email Sender}

	\begin{ucscenarioprincipale}
		\item Lo sviluppatore seleziona la modifica della chiave ``sender'';
		\item Lo sviluppatore inserisce il nuovo valore della proprietà ``sender''.
	\end{ucscenarioprincipale}

	\usecasepost{la proprietà ``sender'' è stata modificata con successo}

	\label{uc:modifica-sender-es}
\end{usecase}

\begin{usecase}{9}{ES}{Modifica proprietà ``password''}
	\usecaseactors{sviluppatore SyncRec}

	\usecasepre{la proprietà ``password'' necessita di essere modificata}

	\usecasedesc{lo sviluppatore vuole modificare la proprietà ``password'' relativa al ``sender'' nel servizio. Il servizio salverà la nuova password criptandola con un opportuno algoritmo}

	\begin{ucscenarioprincipale}
		\item Lo sviluppatore seleziona la modifica della chiave ``password'';
		\item Lo sviluppatore inserisce il nuovo valore della proprietà ``password''.
	\end{ucscenarioprincipale}

	\usecasepost{la proprietà ``password'' è stata modificata con successo}

	\label{uc:modifica-password-es}
\end{usecase}

\begin{usecase}{10}{ES}{Modifica proprietà ``subject''}
	\usecaseactors{sviluppatore SyncRec}

	\usecasepre{la proprietà ``subject'' necessita di essere modificata}

	\usecasedesc{lo sviluppatore vuole modificare la proprietà ``subject'', ossia l'oggetto della email da inviare, nel servizio Email Sender}

	\begin{ucscenarioprincipale}
		\item Lo sviluppatore seleziona la modifica della chiave ``subject'';
		\item Lo sviluppatore inserisce il nuovo valore della proprietà ``subject''.
	\end{ucscenarioprincipale}

	\usecasepost{la proprietà ``subject'' è stata modificata con successo}

	\label{uc:modifica-subject-es}
\end{usecase}

\begin{usecase}{11}{ES}{Invio email}
	\usecaseactors{\textit{front-end}}

	\usecasepre{il \textit{front-end} deve inviare una email in seguito a determinate azioni dell'utente che usa la piattaforma}

	\usecasedesc{il \textit{front-end}, sotto determinate condizioni (come ad esempio la fine della compilazione dello \gls{skill-matrix}), ha necessità di inviare un email automatica specifica al candidato, contenente link e informazioni utili}

	\begin{ucscenarioprincipale}
		\item Il \textit{front-end} invoca le \acrshort{api} per l'invio dell'email.
	\end{ucscenarioprincipale}

	\usecasepost{l'email è stata inviata con successo}

	\begin{ucestensioni}
		\item \ref{uc:vis-errore-invio-es}
	\end{ucestensioni}

	\label{uc:invio-email-es}
\end{usecase}

\begin{usecase}{12}{ES}{Risposta contenente un errore}
	\usecaseactors{\textit{front-end}}

	\usecasepre{una email tenta di essere inviata tramite il servizio Email Sender}

	\usecasedesc{una risposta che spieghi l'errore viene inviata al front-end come risposta se la chiamata dovesse fallire}

	\begin{ucscenarioprincipale}
		\item Il \textit{front-end} riceve la risposta d'errore.
	\end{ucscenarioprincipale}

	\usecasepost{Il messaggio d'errore è stato ricevuto con successo}

	\label{uc:vis-errore-invio-es}
\end{usecase}

% ****************************

\subsection{Casi d'uso relativi al microservizio Login}

\begin{usecase}{1}{L}{Aggiunta nuovo utente}
	\usecaseactors{\textit{front-end}}

	\usecasepre{Un nuovo utente dev'essere inserito nel servizio di Login}

	\usecasedesc{il \textit{front-end} deve inserire un nuovo utente}

	\begin{ucscenarioprincipale}
		\item Il \textit{front-end} inserisce i dati relativi al nuovo utente;
		\item Il \textit{front-end} chiama la funzionalità di inserimento nuovo utente del servizio di Login.
	\end{ucscenarioprincipale}

	\usecasepost{un nuovo utente è stato inserito con successo}

	\begin{ucestensioni}
		\item \ref{uc:vis-errore-ins-utente-l}
	\end{ucestensioni}

	\begin{ucgeneralizzazioni}
		\item \ref{uc:richiesta-aggiunta-utente-l}
	\end{ucgeneralizzazioni}

	\label{uc:aggiunta-utente-l}
\end{usecase}

\begin{usecase}{2}{L}{Aggiunta nuovo utente con richiesta HTTP}
	\usecaseactors{\textit{front-end}}

	\usecasepre{Un nuovo utente dev'essere inserito nel servizio di Login}

	\textbf{\\Descrizione:} il \textit{front-end} deve inserire un nuovo utente. Per fare ciò,
	chiama le opportune \acrshort{api} esposte dal servizio di Login. La richiesta dovrà contenere i campi:
	\begin{itemize}[noitemsep]
		\item username;
		\item email;
		\item password.
	\end{itemize}
	E potrà contenere i campi opzionali:
	\begin{itemize}[noitemsep]
		\item role;
		\item name;
		\item surname;
		\item office.
	\end{itemize}

	\begin{ucscenarioprincipale}
		\item Il \textit{front-end} inserisce i dati relativi al nuovo utente;
		\item Il \textit{front-end} chiama la funzionalità di inserimento nuovo utente del servizio di Login.
	\end{ucscenarioprincipale}

	\usecasepost{un nuovo utente è stato inserito con successo}

	\begin{ucestensioni}
		\item \ref{uc:vis-errore-ins-utente-l}
	\end{ucestensioni}

	\label{uc:richiesta-aggiunta-utente-l}
\end{usecase}

\begin{usecase}{3}{L}{Visualizzazione errore inserimento utente}
	\usecaseactors{\textit{front-end}}

	\usecasepre{c'è un tentativo di aggiunta di un nuovo utente}

	\usecasedesc{viene inviata una risposta al \textit{front-end} contenente un messaggio d'errore per tentativo di inserimento utente non valido}

	\begin{ucscenarioprincipale}
		\item il \textit{front-end} riceve il messaggio d'errore.
	\end{ucscenarioprincipale}

	\usecasepost{il messaggio d'errore è stato ricevuto dal \textit{front-end}}

	\label{uc:vis-errore-ins-utente-l}
\end{usecase}

\begin{usecase}{4}{L}{Modifica utente}
	\usecaseactors{\textit{front-end}}

	\usecasepre{Un utente già presente dev'essere modificato}

	\usecasedesc{il \textit{front-end} deve modificare un utente}

	\begin{ucscenarioprincipale}
		\item Il \textit{front-end} inserisce i dati relativi all'utente da modificare;
		\item Il \textit{front-end} chiama la funzionalità di modifica dell'utente del servizio di Login.
	\end{ucscenarioprincipale}

	\usecasepost{un utente è stato modificato con successo}

	\begin{ucestensioni}
		\item \ref{uc:vis-errore-modifica-utente-l}
	\end{ucestensioni}

	\begin{ucgeneralizzazioni}
		\item \ref{uc:richiesta-modifica-utente-l}
	\end{ucgeneralizzazioni}

	\label{uc:modifica-utente-l}
\end{usecase}

\begin{usecase}{5}{L}{Modifica utente con richiesta HTTP}
	\usecaseactors{\textit{front-end}}

	\usecasepre{Un utente già presente nel servizio di Login dev'essere modificato}

	\textbf{\\Descrizione:} il \textit{front-end} deve modificare un utente. Per fare ciò,
	chiama le opportune \acrshort{api} esposte dal servizio di Login. La richiesta potrà contenere una qualsiasi combinazione dei campi:
	\begin{itemize}[noitemsep]
		\item username;
		\item email;
		\item password;
		\item role;
		\item name;
		\item surname;
		\item office.
	\end{itemize}

	\begin{ucscenarioprincipale}
		\item Il \textit{front-end} inserisce i dati relativi all'utente che vuole modificare;
		\item Il \textit{front-end} chiama le \acrshort{api} di inserimento nuovo utente del servizio di Login.
	\end{ucscenarioprincipale}

	\usecasepost{un utente è stato modificato con successo}

	\begin{ucestensioni}
		\item \ref{uc:vis-errore-modifica-utente-l}
	\end{ucestensioni}

	\label{uc:richiesta-modifica-utente-l}
\end{usecase}

\begin{usecase}{6}{L}{Visualizzazione errore modifica utente}
	\usecaseactors{\textit{front-end}}

	\usecasepre{c'è un tentativo di modifica di un utente}

	\usecasedesc{viene inviata una risposta al \textit{front-end} contenente un messaggio d'errore per tentativo di validazione dei dati relativi all'utente da modificare}

	\begin{ucscenarioprincipale}
		\item il \textit{front-end} riceve il messaggio d'errore.
	\end{ucscenarioprincipale}

	\usecasepost{il messaggio d'errore è stato ricevuto dal \textit{front-end}}

	\label{uc:vis-errore-modifica-utente-l}
\end{usecase}


\begin{usecase}{7}{L}{Rimozione utente}
	\usecaseactors{\textit{front-end}}

	\usecasepre{Un utente già presente dev'essere rimosso dal servizio}

	\usecasedesc{il \textit{front-end} deve rimuovere un utente}

	\begin{ucscenarioprincipale}
		\item Il \textit{front-end} inserisce l'ID relativo all'utente da modificare;
		\item Il \textit{front-end} chiama la funzionalità di rimozione utente del servizio di Login.
	\end{ucscenarioprincipale}

	\usecasepost{un utente è stato rimosso con successo}

	\begin{ucestensioni}
		\item \ref{uc:vis-errore-rimozione-utente-l}
	\end{ucestensioni}

	\begin{ucgeneralizzazioni}
		\item \ref{uc:richiesta-rimozione-utente-l}
	\end{ucgeneralizzazioni}

	\label{uc:rimozione-utente-l}
\end{usecase}

\begin{usecase}{8}{L}{Rimozione utente con richiesta HTTP}
	\usecaseactors{\textit{front-end}}

	\usecasepre{Un utente già presente nel servizio di Login dev'essere rimosso dal servizio}

	\textbf{\\Descrizione:} il \textit{front-end} deve rimuovere un utente. Per fare ciò,
	chiama le opportune \acrshort{api} esposte dal servizio di Login. La richiesta conterrà l'ID dell'utente da eliminare.

	\begin{ucscenarioprincipale}
		\item Il \textit{front-end} inserisce l'ID all'utente che vuole modificare;
		\item Il \textit{front-end} chiama le \acrshort{api} di rimozione utente del servizio di Login.
	\end{ucscenarioprincipale}

	\usecasepost{un utente è stato rimosso dal servizio con successo}

	\begin{ucestensioni}
		\item \ref{uc:vis-errore-rimozione-utente-l}
	\end{ucestensioni}

	\label{uc:richiesta-rimozione-utente-l}
\end{usecase}

\begin{usecase}{9}{L}{Visualizzazione errore rimozione utente}
	\usecaseactors{\textit{front-end}}

	\usecasepre{c'è un tentativo di rimozione utente}

	\usecasedesc{viene inviata una risposta al \textit{front-end} contenente un messaggio d'errore per ID inserito inesistente}

	\begin{ucscenarioprincipale}
		\item il \textit{front-end} riceve il messaggio d'errore.
	\end{ucscenarioprincipale}

	\usecasepost{il messaggio d'errore è stato ricevuto dal \textit{front-end}}

	\label{uc:vis-errore-rimozione-utente-l}
\end{usecase}

\begin{usecase}{10}{L}{Login}
	\usecaseactors{\textit{front-end}}

	\usecasepre{il \textit{front-end} deve effettuare l'operazione di autenticazione per un utente non riconosciuto}

	\textbf{\\Descrizione:} il \textit{front-end} effettua l'operazione di login tramite le \acrshort{api} esposte dal servizio.
	I campi richiesti per l'operazione sono i seguenti:
	\begin{itemize}[noitemsep]
		\item \texttt{username}: può contenere sia il campo ``username'' dell'utente da autenticare sia il campo ``email'';
		\item \texttt{password}.
	\end{itemize}

	\begin{ucscenarioprincipale}
		\item il \textit{front-end} prepara la richiesta con i campi necessari;
		\item il \textit{front-end} effettua l'operazione di login tramite le \acrshort{api} del servizio.
	\end{ucscenarioprincipale}

	\usecasepost{l'operazione di login è avvenuta con successo. Vengono restituiti nella risposta i campi dell'utente autenticato}

	\begin{ucestensioni}
		\item \ref{uc:vis-errore-login-l}
	\end{ucestensioni}

	\label{uc:login-l}
\end{usecase}

\begin{usecase}{11}{L}{Restituzione errore di Login}
	\usecaseactors{\textit{front-end}}

	\usecasepre{viene effettuato un tentativo di login}

	\usecasedesc{viene inviata una risposta al \textit{front-end} contenente un messaggio d'errore relativo al login}

	\begin{ucscenarioprincipale}
		\item il \textit{front-end} riceve il messaggio d'errore.
	\end{ucscenarioprincipale}

	\usecasepost{il messaggio d'errore è stato ricevuto dal \textit{front-end}}

	\begin{ucgeneralizzazioni}
		\item \ref{uc:vis-errore-badrequest-l}
		\item \ref{uc:vis-errore-auth-l}
	\end{ucgeneralizzazioni}

	\label{uc:vis-errore-login-l}

\end{usecase}

\begin{usecase}{12}{L}{Restituzione errore di bad request}
	\usecaseactors{\textit{front-end}}

	\usecasepre{viene effettuato un tentativo di login}

	\usecasedesc{viene inviata una risposta al \textit{front-end} contenente un messaggio d'errore relativo al pessimo tipo di richiesta (\textit{bad request}). Questo può voler dire un tipo di richiesta \acrshort{http} non valida oppure con campi non riconosciuti (e.g. non viene passato il campo ``password'')}

	\begin{ucscenarioprincipale}
		\item il \textit{front-end} riceve il messaggio d'errore riguardante la pessima richiesta.
	\end{ucscenarioprincipale}

	\usecasepost{il messaggio d'errore è stato ricevuto dal \textit{front-end}}

	\label{uc:vis-errore-badrequest-l}

\end{usecase}

\begin{usecase}{13}{L}{Restituzione errore di autenticazione fallita}
	\usecaseactors{\textit{front-end}}

	\usecasepre{viene effettuato un tentativo di login}

	\usecasedesc{viene inviata una risposta al \textit{front-end} contenente un messaggio d'errore relativo all'autenticazione fallita. I campi ``email'' e/o ``password'' non corrispondono a nessun utente registrato}

	\begin{ucscenarioprincipale}
		\item il \textit{front-end} riceve il messaggio d'errore riguardante l'autenticazione fallita.
	\end{ucscenarioprincipale}

	\usecasepost{il messaggio d'errore è stato ricevuto dal \textit{front-end}}

	\label{uc:vis-errore-auth-l}

\end{usecase}

% ****************************

\subsection{Casi d'uso relativi al microservizio Catalog}

\begin{usecase}{1}{C}{Visualizzazione lista skill}
	\usecaseactors{\textit{front-end}}

	\usecasepre{il \textit{front-end} riconosciuto deve visualizzare la lista utenti}

	\usecasedesc{il \textit{front-end} visualizza la lista utenti}

	\begin{ucscenarioprincipale}
		\item Il \textit{front-end} chiama le funzionalità del servizio di Catalog per visualizzare le \textit{skill}.
		\item Il \textit{front-end} visualizza la lista di \textit{skill}.
	\end{ucscenarioprincipale}

	\usecasepost{la lista di \textit{skill} viene visualizzata correttamente}

	\begin{ucgeneralizzazioni}
		\item \ref{uc:vis-skill-http-c}
	\end{ucgeneralizzazioni}

	\label{uc:vis-skill-c}
\end{usecase}

\begin{usecase}{2}{C}{Visualizzazione lista skill tramite richiesta HTTP}
	\usecaseactors{\textit{front-end}}

	\usecasepre{il \textit{front-end} riconosciuto deve visualizzare la lista delle \textit{skill}}

	\usecasedesc{il \textit{front-end} visualizza la lista delle \textit{skill}. Per fare ciò, chiama le opportune \acrshort{http} \acrshort{api} del servizio Catalog}

	\begin{ucscenarioprincipale}
		\item Il \textit{front-end} chiama le \acrshort{api} del servizio di Catalog per la visualizzazione delle \textit{skill}.
		\item Il \textit{front-end} visualizza la lista di \textit{skill}.
	\end{ucscenarioprincipale}

	\usecasepost{la lista di \textit{skill} viene visualizzata correttamente}

	\label{uc:vis-skill-http-c}
\end{usecase}

\begin{usecase}{2.1}{C}{Visualizzazione lista skill per categoria tramite richiesta HTTP}
	\usecaseactors{\textit{front-end}}

	\usecasepre{il \textit{front-end} riconosciuto deve visualizzare la lista delle \textit{skill} appartenenti a una categoria specifica}

	\usecasedesc{il \textit{front-end} visualizza la lista delle \textit{skill} relative a una categoria specificata. Per fare ciò, chiama le opportune \acrshort{http} \acrshort{api} del servizio Catalog}

	\begin{ucscenarioprincipale}
		\item Il \textit{front-end} chiama le \acrshort{api} del servizio di Catalog per la visualizzazione delle \textit{skill} appartenenti a una categoria.
		\item Il \textit{front-end} visualizza la lista di \textit{skill} appartenenti alla categoria di interesse.
	\end{ucscenarioprincipale}

	\usecasepost{la lista di \textit{skill} di una categoria viene visualizzata correttamente}

	\label{uc:vis-skill-category-http-c}
\end{usecase}

\begin{usecase}{3}{C}{Inserimento nuova skill}
	\usecaseactors{Sviluppatore SyncRec}

	\usecasepre{lo sviluppatore deve inserire una nuova \textit{skill} nel servizio Catalog}

	\usecasedesc{lo sviluppatore inserisce una nuova skill nel servizio} %Per fare ciò, chiama le opportune \acrshort{http} \acrshort{api} del servizio Catalog. I campi richiesti per l'aggiunta di una nuova }

	\begin{ucscenarioprincipale}
		\item Lo sviluppatore SyncRec chiama le funzionalità del servizio di Catalog per l'aggiunta di una nuova \textit{skill}.
		\item Lo sviluppatore riceve la risposta di conferma di inserimento.
	\end{ucscenarioprincipale}

	\usecasepost{la \textit{skill} è stata aggiunta correttamente}

	\begin{ucgeneralizzazioni}
		\item \ref{uc:ins-skill-http-c}
	\end{ucgeneralizzazioni}

	\begin{ucestensioni}
		\item \ref{uc:vis-errore-ins-skill-c}
	\end{ucestensioni}

	\label{uc:ins-skill-c}
\end{usecase}

\begin{usecase}{4}{C}{Inserimento nuova skill tramite richiesta HTTP}
	\usecaseactors{Sviluppatore SyncRec}

	\usecasepre{lo sviluppatore deve inserire una nuova \textit{skill} nel servizio Catalog}

	\textbf{\\Descrizione:} lo sviluppatore inserisce una nuova skill nel servizio. Per fare ciò, chiama le opportune \acrshort{http} \acrshort{api} del servizio Catalog. I campi richiesti per l'aggiunta di una nuova skill sono i seguenti:
	\begin{itemize}
		\item \texttt{name}: il nome della \textit{skill};
		\item \texttt{category}: la categoria a cui la \textit{skill} da aggiungere appartiene.
	\end{itemize}

	\begin{ucscenarioprincipale}
		\item Lo sviluppatore SyncRec chiama le funzionalità del servizio di Catalog per l'aggiunta di una nuova \textit{skill}.
		\item Lo sviluppatore riceve la risposta di conferma di inserimento.
	\end{ucscenarioprincipale}

	\usecasepost{la \textit{skill} è stata aggiunta correttamente}

	\begin{ucestensioni}
		\item \ref{uc:vis-errore-ins-skill-c}
	\end{ucestensioni}

	\label{uc:ins-skill-http-c}
\end{usecase}

\begin{usecase}{5}{C}{Visualizzazione errore inserimento skill fallito}
	\usecaseactors{Sviluppatore SyncRec}

	\usecasepre{viene effettuato un tentativo di inserimento \textit{skill}}

	\usecasedesc{viene visualizzato un messaggio d'errore a seguito di un inserimento di una \textit{skill} nel servizio Catalog fallita, dovuto all'assenza di una delle due chiavi richieste (\texttt{name}, \texttt{category})}

	\begin{ucscenarioprincipale}
		\item Lo sviluppatore SyncRec visualizza il messaggio d'errore.
	\end{ucscenarioprincipale}

	\usecasepost{il messaggio d'errore è stato visualizzato dallo sviluppatore}

	\label{uc:vis-errore-ins-skill-c}

\end{usecase}

\begin{usecase}{6}{C}{Modifica skill}
	\usecaseactors{Sviluppatore SyncRec}

	\usecasepre{lo sviluppatore necessita di modificare una \textit{skill} nel servizio Catalog}

	\usecasedesc{lo sviluppatore modifica una \textit{skill} nel servizio Catalog}

	\begin{ucscenarioprincipale}
		\item Lo sviluppatore SyncRec chiama le funzionalità del servizio di Catalog per la modifica di una \textit{skill}.
		\item Lo sviluppatore riceve la risposta di conferma di modifica.
	\end{ucscenarioprincipale}

	\usecasepost{la \textit{skill} è stata modificata correttamente}

	\begin{ucgeneralizzazioni}
		\item \ref{uc:mod-skill-http-c}
	\end{ucgeneralizzazioni}

	\begin{ucestensioni}
		\item \ref{uc:vis-errore-mod-skill-c}
	\end{ucestensioni}

	\label{uc:mod-skill-c}
\end{usecase}

\begin{usecase}{7}{C}{Modifica skill tramite richiesta HTTP}
	\usecaseactors{Sviluppatore SyncRec}

	\usecasepre{lo sviluppatore deve modificare una \textit{skill} presente nel servizio Catalog}

	\textbf{\\Descrizione:} lo sviluppatore modifica una skill del servizio. Per fare ciò, chiama le opportune \acrshort{http} \acrshort{api} del servizio Catalog. I campi richiesti per la modifica di una \textit{skill} possono includere una qualsiasi combinazione tra i seguenti:
	\begin{itemize}
		\item \texttt{name}: il nome della \textit{skill};
		\item \texttt{category}: la categoria a cui la \textit{skill} da aggiungere appartiene.
	\end{itemize}

	\begin{ucscenarioprincipale}
		\item Lo sviluppatore SyncRec chiama le \acrshort{api} del servizio di Catalog per la modifica di una \textit{skill}.
		\item Lo sviluppatore riceve la risposta di conferma di modifica.
	\end{ucscenarioprincipale}

	\usecasepost{la \textit{skill} è stata modificata correttamente}

	\begin{ucestensioni}
		\item \ref{uc:vis-errore-mod-skill-c}
	\end{ucestensioni}

	\label{uc:mod-skill-http-c}
\end{usecase}

\begin{usecase}{8}{C}{Visualizzazione errore modifica skill fallita}
	\usecaseactors{Sviluppatore SyncRec}

	\usecasepre{viene effettuato un tentativo di modifica di una \textit{skill} del servizio}

	\usecasedesc{viene visualizzato un messaggio d'errore a seguito di un tentativo di modifica di una \textit{skill} del servizio Catalog fallito}

	\begin{ucscenarioprincipale}
		\item Lo sviluppatore SyncRec visualizza il messaggio d'errore.
	\end{ucscenarioprincipale}

	\usecasepost{il messaggio d'errore di modifica è stato visualizzato dallo sviluppatore}

	\label{uc:vis-errore-mod-skill-c}

\end{usecase}

\begin{usecase}{9}{C}{Rimozione skill}
	\usecaseactors{Sviluppatore SyncRec}

	\usecasepre{lo sviluppatore necessita di eliminare una \textit{skill} dal servizio Catalog}

	\usecasedesc{lo sviluppatore elimina una \textit{skill} dal servizio Catalog}

	\begin{ucscenarioprincipale}
		\item Lo sviluppatore SyncRec chiama le funzionalità del servizio di Catalog per la rimozione di una \textit{skill}.
		\item Lo sviluppatore riceve la risposta di conferma rimozione.
	\end{ucscenarioprincipale}

	\usecasepost{la \textit{skill} è stata eliminata correttamente}

	\begin{ucgeneralizzazioni}
		\item \ref{uc:rimozione-skill-http-c}
	\end{ucgeneralizzazioni}

	\begin{ucestensioni}
		\item \ref{uc:vis-errore-rimozione-skill-c}
	\end{ucestensioni}

	\label{uc:rimozione-skill-c}
\end{usecase}

\begin{usecase}{10}{C}{Rimozione skill tramite richiesta HTTP}
	\usecaseactors{Sviluppatore SyncRec}

	\usecasepre{lo sviluppatore deve rimuovere una \textit{skill} presente nel servizio Catalog}

	\textbf{\\Descrizione:} lo sviluppatore rimuove una \textit{skill} del servizio. Per fare ciò, chiama le opportune \acrshort{http} \acrshort{api} del servizio Catalog. Per eliminare una \textit{skill}, lo sviluppatore dovrà fornire l'ID relativo a tale \textit{skill}.

	\begin{ucscenarioprincipale}
		\item Lo sviluppatore SyncRec chiama le \acrshort{api} del servizio di Catalog per la rimozione di una \textit{skill}.
		\item Lo sviluppatore riceve la risposta di conferma rimozione.
	\end{ucscenarioprincipale}

	\usecasepost{la \textit{skill} è stata rimossa correttamente}

	\begin{ucestensioni}
		\item \ref{uc:vis-errore-rimozione-skill-c}
	\end{ucestensioni}

	\label{uc:rimozione-skill-http-c}
\end{usecase}

\begin{usecase}{11}{C}{Visualizzazione errore rimozione skill fallita}
	\usecaseactors{Sviluppatore SyncRec}

	\usecasepre{viene effettuato un tentativo di modifica di una \textit{skill} del servizio}

	\usecasedesc{viene visualizzato un messaggio d'errore a seguito di un tentativo di eliminazione di una \textit{skill} del servizio Catalog fallito, dovuto a un ID non riconosciuto}

	\begin{ucscenarioprincipale}
		\item Lo sviluppatore SyncRec visualizza il messaggio d'errore.
	\end{ucscenarioprincipale}

	\usecasepost{il messaggio d'errore di rimozione è stato visualizzato dallo sviluppatore}

	\label{uc:vis-errore-rimozione-skill-c}

\end{usecase}

% ***********************************

\subsection{Casi d'uso relativi al microservizio Applicant}

\begin{usecase}{1}{A}{Visualizzazione lista applicant}
	\usecaseactors{\textit{front-end}}

	\usecasepre{il \textit{front-end} riconosciuto deve visualizzare la lista degli \textit{applicant}}

	\usecasedesc{il \textit{front-end} visualizza la lista \textit{applicant}}

	\begin{ucscenarioprincipale}
		\item Il \textit{front-end} chiama le funzionalità del servizio di Applicant per visualizzare la lista di tutti i candidati.
		\item Il \textit{front-end} visualizza la lista degli \textit{applicant}.
	\end{ucscenarioprincipale}

	\usecasepost{la lista degli \textit{applicant} viene visualizzata correttamente}

	%	\begin{ucestensioni}
	%		\item UC3-L
	%	\end{ucestensioni}

	\begin{ucgeneralizzazioni}
		\item \ref{uc:vis-lista-applicant-http-a}
	\end{ucgeneralizzazioni}

	\label{uc:vis-lista-applicant-a}
\end{usecase}

\begin{usecase}{2}{A}{Visualizzazione lista applicant tramite richiesta HTTP}
	\usecaseactors{\textit{front-end}}

	\usecasepre{il \textit{front-end} riconosciuto deve visualizzare la lista degli \textit{applicant}}

	\usecasedesc{il \textit{front-end} visualizza la lista degli \textit{applicant}. Per fare ciò, chiama le opportune \acrshort{http} \acrshort{api} del servizio Applicant}

	\begin{ucscenarioprincipale}
		\item Il \textit{front-end} chiama le \acrshort{api} del servizio di Applicant per la visualizzazione dei candidati.
		\item Il \textit{front-end} visualizza la lista dei candidati.
	\end{ucscenarioprincipale}

	\usecasepost{la lista degli \textit{applicant} viene visualizzata correttamente}

	\label{uc:vis-lista-applicant-http-a}
\end{usecase}

\begin{usecase}{2.1}{A}{Visualizzazione lista applicant considerabili tramite richiesta HTTP}
	\usecaseactors{\textit{front-end}}

	\usecasepre{il \textit{front-end} riconosciuto deve visualizzare la lista degli \textit{applicant} considerabili}

	\usecasedesc{il \textit{front-end} visualizza la lista degli \textit{applicant} marcati come considerabili. Per fare ciò, chiama le opportune \acrshort{http} \acrshort{api} del servizio Applicant. Un \textit{applicant} è considerato considerabile se il suo campo \texttt{considerable} è \texttt{true}}

	\begin{ucscenarioprincipale}
		\item Il \textit{front-end} chiama le \acrshort{api} del servizio Applicant per la visualizzazione degli \textit{applicant} considerabili.
		\item Il \textit{front-end} visualizza la lista degli \textit{applicant} considerabili.
	\end{ucscenarioprincipale}

	\usecasepost{la lista degli \textit{applicant} considerabili viene visualizzata correttamente}

	\label{uc:vis-lista-applicant-considerable-http-a}
\end{usecase}

\begin{usecase}{2.2}{A}{Visualizzazione lista applicant per seniority tramite richiesta HTTP}
	\usecaseactors{\textit{front-end}}

	\usecasepre{il \textit{front-end} riconosciuto deve visualizzare la lista degli \textit{applicant} appartenenti a una \textit{seniority}}

	\textbf{\\Descrizione:} il \textit{front-end} visualizza la lista degli \textit{applicant} appartenenti a una \textit{seniority} specifica. Per fare ciò, chiama le opportune \acrshort{http} \acrshort{api} del servizio Applicant. La \textit{seniority} di interesse verrà passata come parametro insieme alla richiesta.
	I tipi di \textit{seniority} sono:
	\begin{ucitemize}
		\item Junior;
		\item Intermediate;
		\item Senior.
	\end{ucitemize}

	\begin{ucscenarioprincipale}
		\item Il \textit{front-end} chiama le \acrshort{api} del servizio Applicant per la visualizzazione degli \textit{applicant}.
		\item Il \textit{front-end} visualizza la lista degli \textit{applicant} con la \textit{seniority} specificata.
	\end{ucscenarioprincipale}

	\usecasepost{la lista degli \textit{applicant} con la \textit{seniority} di interesse viene visualizzata correttamente}

	\label{uc:vis-lista-applicant-seniority-http-a}
\end{usecase}

\begin{usecase}{2.3}{A}{Visualizzazione lista applicant per scope tramite richiesta HTTP}
	\usecaseactors{\textit{front-end}}

	\usecasepre{il \textit{front-end} riconosciuto deve visualizzare la lista degli \textit{applicant} appartenenti a uno \textit{scope} specifico}

	\textbf{\\Descrizione:} il \textit{front-end} visualizza la lista degli \textit{applicant} appartenenti a uno \textit{scope} specifico. Per fare ciò, chiama le opportune \acrshort{http} \acrshort{api} del servizio Applicant. Lo \textit{scope} di interesse verrà passato come parametro insieme alla richiesta.
	I tipi di \textit{scope} sono:
	\begin{ucitemize}
		\item Developer;
		\item System Engineer;
		\item Office Worker.
	\end{ucitemize}

	\begin{ucscenarioprincipale}
		\item Il \textit{front-end} chiama le \acrshort{api} del servizio Applicant per la visualizzazione degli \textit{applicant}.
		\item Il \textit{front-end} visualizza la lista degli \textit{applicant} con lo \textit{scope} specificato.
	\end{ucscenarioprincipale}

	\usecasepost{la lista degli \textit{applicant} con la \textit{seniority} di interesse viene visualizzata correttamente}

	\label{uc:vis-lista-applicant-scope-http-a}
\end{usecase}

\begin{usecase}{3}{A}{Visualizzazione applicant}
	\usecaseactors{\textit{front-end}}

	\usecasepre{il \textit{front-end} riconosciuto deve visualizzare un \textit{applicant} specifico}

	\usecasedesc{il \textit{front-end} visualizza l'\textit{applicant}}

	\begin{ucscenarioprincipale}
		\item Il \textit{front-end} chiama le funzionalità del servizio di Applicant per la visualizzazione.
		\item Il \textit{front-end} visualizza l'\textit{applicant} di interesse.
	\end{ucscenarioprincipale}

	\usecasepost{l'\textit{applicant} viene visualizzato correttamente}

	\begin{ucgeneralizzazioni}
		\item \ref{uc:vis-applicant-http-a}
	\end{ucgeneralizzazioni}

	\label{uc:vis-applicant-a}
\end{usecase}

\begin{usecase}{4}{A}{Visualizzazione applicant tramite richiesta HTTP}
	\usecaseactors{\textit{front-end}}

	\usecasepre{il \textit{front-end} riconosciuto deve visualizzare un \textit{applicant}}

	\usecasedesc{il \textit{front-end} visualizza l'\textit{applicant}. Per fare ciò, chiama le opportune \acrshort{http} \acrshort{api} del servizio Applicant}

	\begin{ucscenarioprincipale}
		\item Il \textit{front-end} chiama le \acrshort{api} del servizio di Applicant per la visualizzazione.
		\item Il \textit{front-end} visualizza il candidato di interesse.
	\end{ucscenarioprincipale}

	\begin{ucestensioni}
		\item \ref{uc:vis-errore-applicant-a}
	\end{ucestensioni}

	\usecasepost{l'\textit{applicant} viene visualizzato correttamente}

	\label{uc:vis-applicant-http-a}
\end{usecase}

\begin{usecase}{4.1}{A}{Visualizzazione applicant per ID tramite richiesta HTTP}
	\usecaseactors{\textit{front-end}}

	\usecasepre{il \textit{front-end} riconosciuto deve visualizzare un \textit{applicant}}

	\usecasedesc{il \textit{front-end} visualizza l'\textit{applicant} con un ID specifico. Per fare ciò, chiama le opportune \acrshort{http} \acrshort{api} del servizio Applicant, passando nella richiesta il parametro relativo all'ID}

	\begin{ucscenarioprincipale}
		\item Il \textit{front-end} chiama le \acrshort{api} del servizio di Applicant per la visualizzazione.
		\item Il \textit{front-end} visualizza il candidato con l'ID specificato.
	\end{ucscenarioprincipale}

	\begin{ucestensioni}
		\item \ref{uc:vis-errore-applicant-a}
	\end{ucestensioni}

	\usecasepost{l'\textit{applicant} con l'ID specificato viene visualizzato correttamente}

	\label{uc:vis-applicant-id-http-a}
\end{usecase}

\begin{usecase}{4.2}{A}{Visualizzazione applicant per email tramite richiesta HTTP}
	\usecaseactors{\textit{front-end}}

	\usecasepre{il \textit{front-end} riconosciuto deve visualizzare un \textit{applicant}}

	\usecasedesc{il \textit{front-end} visualizza l'\textit{applicant} per email. Per fare ciò, chiama le opportune \acrshort{http} \acrshort{api} del servizio Applicant, passando nella richiesta il parametro relativo all'email}

	\begin{ucscenarioprincipale}
		\item Il \textit{front-end} chiama le \acrshort{api} del servizio di Applicant per la visualizzazione.
		\item Il \textit{front-end} visualizza il candidato con l'email specificato.
	\end{ucscenarioprincipale}

	\begin{ucestensioni}
		\item \ref{uc:vis-errore-applicant-a}
	\end{ucestensioni}

	\usecasepost{l'\textit{applicant} con l'email specificata viene visualizzato correttamente}

	\label{uc:vis-applicant-email-http-a}
\end{usecase}

\begin{usecase}{4.3}{A}{Visualizzazione applicant per numero di telefono tramite richiesta HTTP}
	\usecaseactors{\textit{front-end}}

	\usecasepre{il \textit{front-end} riconosciuto deve visualizzare un \textit{applicant}}

	\usecasedesc{il \textit{front-end} visualizza l'\textit{applicant} con un numero di telefono specifico. Per fare ciò, chiama le opportune \acrshort{http} \acrshort{api} del servizio Applicant, passando nella richiesta il parametro \texttt{phone}, relativo al numero di telefono}

	\begin{ucscenarioprincipale}
		\item Il \textit{front-end} chiama le \acrshort{api} del servizio di Applicant per la visualizzazione.
		\item Il \textit{front-end} visualizza il candidato con il numero di telefono specificato.
	\end{ucscenarioprincipale}

	\begin{ucestensioni}
		\item \ref{uc:vis-errore-applicant-a}
	\end{ucestensioni}

	\usecasepost{l'\textit{applicant} con il numero di telefono specificato viene visualizzato correttamente}

	\label{uc:vis-applicant-phone-http-a}
\end{usecase}


\begin{usecase}{5}{A}{Visualizzazione errore applicant non trovato}
	\usecaseactors{\textit{front-end}}

	\usecasepre{viene effettuato una ricerca di un applicant specifico}

	\usecasedesc{viene restituito un messaggio d'errore dal servizio a seguito del tentativo di visualizzazione di un \textit{applicant} specifico}

	\begin{ucscenarioprincipale}
		\item Il \textit{front-end} riceve il messaggio d'errore.
	\end{ucscenarioprincipale}

	\usecasepost{il messaggio d'errore è stato restituito e visualizzato dal \textit{front-end}}

	\label{uc:vis-errore-applicant-a}

\end{usecase}

\begin{usecase}{6}{A}{Aggiunta nuovo applicant}
	\usecaseactors{\textit{front-end}}

	\usecasepre{un nuovo \textit{applicant} dev'essere inserito nel servizio Applicant}

	\usecasedesc{il \textit{front-end} deve inserire un nuovo \textit{applicant}}

	\begin{ucscenarioprincipale}
		\item Il \textit{front-end} inserisce i dati relativi al nuovo \textit{applicant};
		\item Il \textit{front-end} chiama la funzionalità di inserimento nuovo \textit{applicant}.
	\end{ucscenarioprincipale}

	\usecasepost{un nuovo \textit{applicant} è stato inserito con successo}

	\begin{ucestensioni}
		\item \ref{uc:vis-errore-ins-applicant-a}
	\end{ucestensioni}

	\begin{ucgeneralizzazioni}
		\item \ref{uc:richiesta-aggiunta-applicant-a}
	\end{ucgeneralizzazioni}

	\label{uc:aggiunta-applicant-a}
\end{usecase}

\begin{usecase}{7}{A}{Aggiunta nuovo applicant con richiesta HTTP}
	\usecaseactors{\textit{front-end}}

	\usecasepre{Un nuovo \textit{applicant} dev'essere inserito nel servizio Applicant}

	\textbf{\\Descrizione:} il \textit{front-end} deve inserire un nuovo \textit{applicant}. Per fare ciò,
	chiama le opportune \acrshort{api} esposte dal servizio Applicant. La richiesta dovrà contenere i campi:
	\begin{itemize}[noitemsep]
		\item \texttt{name};
		\item \texttt{surname};
		\item \texttt{email};
		\item \texttt{phone}.
	\end{itemize}
	E potrà contenere i campi opzionali:
	\begin{ucitemize}
		\item \texttt{country};
		\item \texttt{address};
		\item \texttt{birthday};
		\item \texttt{notes};
		\item \texttt{jobStatus};
		\item \texttt{interviews};
		\item \texttt{considerable};
		\item \texttt{curriculumId};
		\item \texttt{ratingId};
		\item \texttt{scope};
		\item \texttt{seniority};
		\item \texttt{genre};
		\item \texttt{skills}.
	\end{ucitemize}

	\begin{ucscenarioprincipale}
		\item Il \textit{front-end} inserisce i dati relativi al nuovo \textit{applicant};
		\item Il \textit{front-end} chiama la funzionalità di inserimento nuovo \textit{applicant} del servizio.
	\end{ucscenarioprincipale}

	\usecasepost{un nuovo \textit{applicant} è stato inserito con successo}

	\begin{ucestensioni}
		\item \ref{uc:vis-errore-ins-applicant-a}
	\end{ucestensioni}

	\label{uc:richiesta-aggiunta-applicant-a}
\end{usecase}

\begin{usecase}{8}{A}{Visualizzazione errore inserimento applicant}
	\usecaseactors{\textit{front-end}}

	\usecasepre{c'è un tentativo di aggiunta di un nuovo \textit{applicant}}

	\usecasedesc{viene inviata una risposta al \textit{front-end} contenente un messaggio d'errore per tentativo di inserimento \textit{applicant} non valido}

	\begin{ucscenarioprincipale}
		\item il \textit{front-end} riceve il messaggio d'errore.
	\end{ucscenarioprincipale}

	\usecasepost{il messaggio d'errore è stato ricevuto dal \textit{front-end}}

	\label{uc:vis-errore-ins-applicant-a}
\end{usecase}

\begin{usecase}{9}{A}{Modifica applicant}
	\usecaseactors{\textit{front-end}}

	\usecasepre{un \textit{applicant} già presente dev'essere modificato}

	\usecasedesc{il \textit{front-end} modifica un \textit{applicant} presente nel servizio Applicant}

	\begin{ucscenarioprincipale}
		\item Il \textit{front-end} inserisce i dati modificati dell'\textit{applicant} presente;
		\item Il \textit{front-end} chiama la funzionalità di modifica \textit{applicant}.
	\end{ucscenarioprincipale}

	\usecasepost{un \textit{applicant} è stato modificato con successo}

	\begin{ucestensioni}
		\item \ref{uc:vis-errore-mod-applicant-a}
	\end{ucestensioni}

	\begin{ucgeneralizzazioni}
		\item \ref{uc:richiesta-mod-applicant-a}
	\end{ucgeneralizzazioni}

	\label{uc:mod-applicant-a}
\end{usecase}

\begin{usecase}{10}{A}{Modifica applicant con richiesta HTTP}
	\usecaseactors{\textit{front-end}}

	\usecasepre{un \textit{applicant} già presente dev'essere modificato}

	\textbf{\\Descrizione:} il \textit{front-end} modifica un \textit{applicant} presente nel servizio. Per fare ciò,
	chiama le opportune \acrshort{api} esposte dal servizio Applicant. La richiesta potrà contenere una qualsiasi combinazione dei campi:
	\begin{ucitemize}
		\item \texttt{name};
		\item \texttt{surname};
		\item \texttt{email};
		\item \texttt{phone}.
		\item \texttt{country};
		\item \texttt{address};
		\item \texttt{birthday};
		\item \texttt{notes};
		\item \texttt{jobStatus};
		\item \texttt{interviews};
		\item \texttt{considerable};
		\item \texttt{curriculumId};
		\item \texttt{ratingId};
		\item \texttt{scope};
		\item \texttt{seniority};
		\item \texttt{genre};
		\item \texttt{skills}.
	\end{ucitemize}

	\begin{ucscenarioprincipale}
		\item Il \textit{front-end} inserisce i dati relativi all'\textit{applicant} da modificare;
		\item Il \textit{front-end} chiama la funzionalità di modifica \textit{applicant} del servizio.
	\end{ucscenarioprincipale}

	\usecasepost{un \textit{applicant} è stato modificato con successo}

	\begin{ucestensioni}
		\item \ref{uc:vis-errore-mod-applicant-a}
	\end{ucestensioni}

	\label{uc:richiesta-mod-applicant-a}
\end{usecase}

\begin{usecase}{11}{A}{Visualizzazione errore modifica applicant}
	\usecaseactors{\textit{front-end}}

	\usecasepre{c'è un tentativo di modifica di un \textit{applicant}}

	\usecasedesc{viene inviata una risposta al \textit{front-end} contenente un messaggio d'errore per tentativo di modifica \textit{applicant} non valido}

	\begin{ucscenarioprincipale}
		\item il \textit{front-end} riceve il messaggio d'errore.
	\end{ucscenarioprincipale}

	\usecasepost{il messaggio d'errore è stato ricevuto dal \textit{front-end}}

	\label{uc:vis-errore-mod-applicant-a}
\end{usecase}


\begin{usecase}{12}{A}{Rimozione applicant}
	\usecaseactors{\textit{front-end}}

	\usecasepre{un \textit{applicant} già presente dev'essere rimosso dal servizio}

	\usecasedesc{il \textit{front-end} rimuove un \textit{applicant} presente nel servizio Applicant}

	\begin{ucscenarioprincipale}
		\item Il \textit{front-end} chiama la funzionalità di rimozione \textit{applicant}.
	\end{ucscenarioprincipale}

	\usecasepost{un \textit{applicant} è stato rimosso con successo}

	\begin{ucestensioni}
		\item \ref{uc:vis-errore-rimozione-applicant-a}
	\end{ucestensioni}

	\begin{ucgeneralizzazioni}
		\item \ref{uc:richiesta-rimozione-applicant-a}
	\end{ucgeneralizzazioni}

	\label{uc:rimozione-applicant-a}
\end{usecase}

\begin{usecase}{13}{A}{Rimozione applicant con richiesta HTTP}
	\usecaseactors{\textit{front-end}}

	\usecasepre{un \textit{applicant} già presente dev'essere modificato}

	\textbf{\\Descrizione:} il \textit{front-end} rimuove un \textit{applicant} presente nel servizio. Per fare ciò,
	chiama le opportune \acrshort{api} esposte dal servizio Applicant. La richiesta dovrà contenere l'ID dell'\textit{applicant} da rimuovere.

	\begin{ucscenarioprincipale}
		\item Il \textit{front-end} chiama la funzionalità di rimozione di un \textit{applicant} del servizio.
	\end{ucscenarioprincipale}

	\usecasepost{un \textit{applicant} è stato rimosso con successo}

	\begin{ucestensioni}
		\item \ref{uc:vis-errore-rimozione-applicant-a}
	\end{ucestensioni}

	\label{uc:richiesta-rimozione-applicant-a}
\end{usecase}

\begin{usecase}{14}{A}{Visualizzazione errore rimozione applicant}
	\usecaseactors{\textit{front-end}}

	\usecasepre{c'è un tentativo di rimozione di un \textit{applicant}}

	\usecasedesc{viene inviata una risposta al \textit{front-end} contenente un messaggio d'errore per tentativo di rimozione \textit{applicant} non valido (e.g. l'ID non esiste)}

	\begin{ucscenarioprincipale}
		\item il \textit{front-end} riceve il messaggio d'errore.
	\end{ucscenarioprincipale}

	\usecasepost{il messaggio d'errore è stato ricevuto dal \textit{front-end}}

	\label{uc:vis-errore-rimozione-applicant-a}
\end{usecase}

% ****************************


% ****************************
% ****************************
% ****************************

\section{Tracciamento dei requisiti}

Da un'attenta analisi dei requisiti e degli use case effettuata sul progetto SyncRec è stata stilata la tabella che traccia i requisiti in rapporto agli \textit{use case}.

Sono stati individuati diversi tipi di requisiti e si è quindi fatto utilizzo di un codice identificativo per distinguerli.

%Il codice dei requisiti è così strutturato R(F/Q/V)(N/D/O) dove:
%\begin{enumerate}
%	\item[R =] requisito
%    \item[F =] funzionale
%    \item[Q =] qualitativo
%    \item[V =] di vincolo
%    \item[N =] obbligatorio (necessario)
%    \item[D =] desiderabile
%    \item[Z =] opzionale
%\end{enumerate}
%Nelle tabelle \ref{tab:requisiti-funzionali}, \ref{tab:requisiti-qualitativi} e \ref{tab:requisiti-vincolo} sono riassunti i requisiti e il loro tracciamento con gli use case delineati in fase di analisi.

%Ad ogni requisito viene assegnato il codice identificativo univoco:
\begin{center}
	\texttt{R[Tipo][Priorità]-[Numero]}
\end{center}
in cui ogni parte ha un significato preciso:
\begin{itemize}
	\item \textbf{R}: requisito;
	\item \textbf{Tipo}: la la tipologia di requisito che può essere di:
	\begin{itemize}
		\item \textbf{F}: funzionalità;
		\item \textbf{Q}: qualità;
		\item \textbf{V}: vincolo.
	\end{itemize}
	\item \textbf{Priorità}: indica il grado di urgenza di un requisito di essere soddisfatto, come:
	\begin{itemize}
		\item \textbf{O}: obbligatorio;
		\item \textbf{D}: desiderabile;
		\item \textbf{F}: facoltativo.
	\end{itemize}
	\item \textbf{Numero}: numero progressivo che segue la struttura dei documenti;
\end{itemize}

Esempio: \texttt{RFD-3} indica il terzo requisito di funzionalità ed è desiderabile.

Nelle tabelle \ref{tab:requisiti-funzionali-es}, \ref{tab:requisiti-qualitativi} e \ref{tab:requisiti-vincolo} sono riassunti i requisiti e il loro tracciamento con gli use case delineati in fase di analisi.

\begin{table}[H]%
	\begin{paddedtablex}[1.4]{\textwidth}{cXc}

		\textbf{Requisito} & \textbf{Descrizione} & \textbf{Use Case}\\
		\toprule

		\reqrow{RFO}{Il servizio Email Sender permette di aggiungere una nuova proprietà}{\ref{uc:aggiunta-proprieta-es}}
		\reqrow{RFD}{Il servizio Email Sender deve restituire un messaggio d'errore in caso un inserimento di proprietà non vada a buon fine}{\ref{uc:vis-errore-ins-proprieta-es}}
		\reqrow{RFO}{Il servizio Email Sender permette di aggiungere la proprietà ``sender''}{\ref{uc:aggiunta-sender-es}}
		\reqrow{RFO}{Il servizio Email Sender permette di aggiungere la proprietà ``password''}{\ref{uc:aggiunta-password-es}}
		\reqrow{RFO}{Il servizio Email Sender permette di aggiungere la proprietà ``subject''}{\ref{uc:aggiunta-subject-es}}
		\reqrow{RFO}{Il servizio Email Sender permette di modificare una proprietà già presente}{\ref{uc:modifica-proprieta-es}}
		\reqrow{RFD}{Il servizio Email Sender deve restituire un messaggio d'errore in caso la modifica di proprietà fallisca}{\ref{uc:vis-errore-mod-proprieta-es}}
		\reqrow{RFO}{Il servizio Email Sender permette di modificare la proprietà ``sender''}{\ref{uc:modifica-sender-es}}
		\reqrow{RFO}{Il servizio Email Sender permette di modificare la proprietà ``password''}{\ref{uc:modifica-password-es}}
		\reqrow{RFO}{Il servizio Email Sender permette di modificare la proprietà ``subject''}{\ref{uc:modifica-subject-es}}
		\reqrow{RFO}{Il servizio Email Sender permette di inviare un'email usando le proprietà ``sender'', ``password'', ``subject''}{\ref{uc:invio-email-es}}
		\reqrow{RFD}{Il servizio Email Sender deve restituire un messaggio d'errore in caso l'invio email fallisca}{\ref{uc:vis-errore-invio-es}}

		\bottomrule
	\end{paddedtablex}
	\vspace{4pt}
	\caption{Tabella del tracciamento dei requisti funzionali (Email Sender service)}
	\label{tab:requisiti-funzionali-es}
\end{table}%

\begin{table}[H]%
	\begin{paddedtablex}[1.4]{\textwidth}{cXc}

		\textbf{Requisito} & \textbf{Descrizione} & \textbf{Use Case}\\
		\toprule

		\reqrow{RFO}{Il servizio Login permette di inserire un nuovo utente}{\ref{uc:aggiunta-utente-l}}
		\reqrow{RFO}{Il servizio Login permette di inserire un nuovo utente tramite richiesta HTTP}{\ref{uc:richiesta-aggiunta-utente-l}}
		\reqrow{RFD}{Il servizio Login restituisce un errore in caso l'aggiunta utente fallisca}{\ref{uc:vis-errore-ins-utente-l}}

		\reqrow{RFO}{Il servizio Login permette di modificare un utente già presente}{\ref{uc:modifica-utente-l}}
		\reqrow{RFO}{Il servizio Login permette di modificare un utente tramite richiesta HTTP}{\ref{uc:richiesta-modifica-utente-l}}
		\reqrow{RFD}{Il servizio Login restituisce un errore in caso la modifica utente fallisca}{\ref{uc:vis-errore-modifica-utente-l}}

		\reqrow{RFO}{Il servizio Login permette di rimuovere un utente già presente con ID specificato}{\ref{uc:rimozione-utente-l}}
		\reqrow{RFO}{Il servizio Login permette di rimuovere un utente con ID specificato tramite richiesta HTTP}{\ref{uc:richiesta-rimozione-utente-l}}
		\reqrow{RFD}{Il servizio Login restituisce un errore in caso la rimozione utente fallisca}{\ref{uc:vis-errore-rimozione-utente-l}}

		\reqrow{RFO}{Il servizio Login permette di effettuare l'operazione di login}{\ref{uc:login-l}}
		\reqrow{RFO}{Il servizio Login restituisce un errore specifico in caso l'operazione di login fallisca}{\ref{uc:vis-errore-login-l}}
		\reqrow{RFO}{Il servizio Login restituisce un errore specifico in caso l'operazione di login fallisca per una richiesta malformata}{\ref{uc:vis-errore-badrequest-l}}
		\reqrow{RFO}{Il servizio Login restituisce un errore specifico in caso l'operazione di login fallisca per un utenza non riconosciuta}{\ref{uc:vis-errore-auth-l}}

		\bottomrule
	\end{paddedtablex}
	\vspace{4pt}
	\caption{Tabella del tracciamento dei requisti funzionali (Login service)}
	\label{tab:requisiti-funzionali-l}
\end{table}%

\begin{table}[H]%
	\begin{paddedtablex}[1.4]{\textwidth}{cXc}

		\textbf{Requisito} & \textbf{Descrizione} & \textbf{Use Case}\\
		\toprule

		\reqrow{RFO}{Il servizio Catalog permette di visualizzare la lista delle skill}{\ref{uc:vis-skill-c}}
		\reqrow{RFO}{Il servizio Catalog permette di visualizzare la lista delle skill tramite richiesta HTTP}{\ref{uc:vis-skill-http-c}}
		\reqrow{RFD}{Il servizio Catalog permette di visualizzare la lista delle skill appartenenti a una specifica categoria tramite richiesta HTTP}{\ref{uc:vis-skill-category-http-c}}

		\reqrow{RFO}{Il servizio Catalog permette di inserire una nuova skill}{\ref{uc:ins-skill-c}}
		\reqrow{RFO}{Il servizio Catalog permette di inserire una nuova skill tramite richiesta HTTP}{\ref{uc:ins-skill-http-c}}
		\reqrow{RFD}{Il servizio Catalog restituisce un errore in caso l'aggiunta della skill fallisca}{\ref{uc:vis-errore-ins-skill-c}}

		\reqrow{RFO}{Il servizio Catalog permette di modificare una skill già presente}{\ref{uc:mod-skill-c}}
		\reqrow{RFO}{Il servizio Catalog permette di modificare una skill tramite richiesta HTTP}{\ref{uc:mod-skill-http-c}}
		\reqrow{RFD}{Il servizio Catalog restituisce un errore in caso la modifica di una skill fallisca}{\ref{uc:vis-errore-mod-skill-c}}

		\reqrow{RFO}{Il servizio Catalog permette di rimuovere una skill con ID specifico già presente}{\ref{uc:rimozione-skill-c}}
		\reqrow{RFO}{Il servizio Catalog permette di rimuovere una skill con ID specifico tramite richiesta HTTP}{\ref{uc:rimozione-skill-http-c}}
		\reqrow{RFD}{Il servizio Catalog restituisce un errore in caso la rimozione di una skill fallisca}{\ref{uc:vis-errore-rimozione-skill-c}}

		\bottomrule
	\end{paddedtablex}
	\vspace{4pt}
	\caption{Tabella del tracciamento dei requisti funzionali (Catalog service)}
	\label{tab:requisiti-funzionali-c}
\end{table}%


\begin{table}[H]%
	\begin{paddedtablex}[1.4]{\textwidth}{cXc}

		\textbf{Requisito} & \textbf{Descrizione} & \textbf{Use Case}\\
		\toprule

		\reqrow{RFO}{Il servizio Applicant permette di visualizzare la lista degli applicant}{\ref{uc:vis-lista-applicant-a}}
		\reqrow{RFO}{Il servizio Applicant permette di visualizzare la lista degli applicant tramite richiesta HTTP}{\ref{uc:vis-lista-applicant-http-a}}
		\reqrow{RFD}{Il servizio Applicant permette di visualizzare la lista degli applicant marcati come considerabili tramite richiesta HTTP}{\ref{uc:vis-lista-applicant-considerable-http-a}}
		\reqrow{RFD}{Il servizio Applicant permette di visualizzare la lista degli applicant con una specifica ``seniority'' tramite richiesta HTTP}{\ref{uc:vis-lista-applicant-seniority-http-a}}
		\reqrow{RFD}{Il servizio Applicant permette di visualizzare la lista degli applicant con uno specifico ``scope'' tramite richiesta HTTP}{\ref{uc:vis-lista-applicant-scope-http-a}}

		\reqrow{RFO}{Il servizio Applicant permette di visualizzare un applicant specifico}{\ref{uc:vis-applicant-a}}
		\reqrow{RFO}{Il servizio Applicant permette di visualizzare un applicant specifico tramite richiesta HTTP}{\ref{uc:vis-applicant-http-a}}
		\reqrow{RFO}{Il servizio Applicant permette di visualizzare un applicant con un ID specifico tramite richiesta HTTP}{\ref{uc:vis-applicant-id-http-a}}
		\reqrow{RFD}{Il servizio Applicant permette di visualizzare un applicant con un email specifica tramite richiesta HTTP}{\ref{uc:vis-applicant-email-http-a}}
		\reqrow{RFD}{Il servizio Applicant permette di visualizzare un applicant con un numero di telefono specificato tramite richiesta HTTP}{\ref{uc:vis-applicant-phone-http-a}}
		\reqrow{RFF}{Il servizio Applicant mostra un opportuno errore nel caso un applicant ricercato non sia presente}{\ref{uc:vis-errore-applicant-a}}

		\reqrow{RFO}{Il servizio Applicant permette di inserire un nuovo applicant}{\ref{uc:aggiunta-applicant-a}}
		\reqrow{RFO}{Il servizio Applicant permette di inserire un nuovo applicant tramite richiesta HTTP}{\ref{uc:richiesta-aggiunta-applicant-a}}
		\reqrow{RFD}{Il servizio Applicant restituisce un errore in caso l'aggiunta dell'applicant fallisca}{\ref{uc:vis-errore-ins-applicant-a}}

		\reqrow{RFO}{Il servizio Applicant permette di modificare  un applicant già presente}{\ref{uc:mod-applicant-a}}
		\reqrow{RFO}{Il servizio Applicant permette di modifica un applicant tramite richiesta HTTP}{\ref{uc:richiesta-mod-applicant-a}}
		\reqrow{RFD}{Il servizio Applicant restituisce un errore in caso la modifica di un applicant fallisca}{\ref{uc:vis-errore-mod-applicant-a}}

		\reqrow{RFO}{Il servizio Applicant permette di rimuovere un applicant già presente}{\ref{uc:rimozione-applicant-a}}
		\reqrow{RFO}{Il servizio Applicant permette di rimuovere un applicant tramite richiesta HTTP}{\ref{uc:richiesta-rimozione-applicant-a}}
		\reqrow{RFD}{Il servizio Applicant restituisce un errore in caso la rimozione dell'applicant fallisca}{\ref{uc:vis-errore-rimozione-applicant-a}}

		\bottomrule
	\end{paddedtablex}
	\vspace{4pt}
	\caption{Tabella del tracciamento dei requisti funzionali (Applicant service)}
	\label{tab:requisiti-funzionali-a}
\end{table}%

%\vspace{5cm}

\begin{table}[H]%
\caption{Tabella del tracciamento dei requisiti qualitativi}
\label{tab:requisiti-qualitativi}
\begin{tabularx}{\textwidth}{lXl}
\hline\hline
\textbf{Requisito} & \textbf{Descrizione} & \textbf{Use Case}\\
\hline
RQD-1    & Le prestazioni del simulatore hardware deve garantire la giusta esecuzione dei test e non la generazione di falsi negativi & - \\
\hline
\end{tabularx}
\end{table}%

\begin{table}[H]%
\caption{Tabella del tracciamento dei requisiti di vincolo}
\label{tab:requisiti-vincolo}
\begin{tabularx}{\textwidth}{lXl}
\hline\hline
\textbf{Requisito} & \textbf{Descrizione} & \textbf{Use Case}\\
\hline
RVO-1    & La libreria per l'esecuzione dei test automatici deve essere riutilizzabile & - \\
\hline
\end{tabularx}
\end{table}%