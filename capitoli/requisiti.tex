% !TEX encoding = UTF-8
% !TEX TS-program = pdflatex
% !TEX root = ../tesi.tex

%**************************************************************
\chapter{Analisi dei requisiti}
\label{cap:analisi-requisiti}
%**************************************************************

\intro{Breve introduzione al capitolo}\\ % TODO

\section{Casi d'uso}

Per lo studio dei casi di utilizzo del prodotto sono stati creati dei diagrammi.
I diagrammi dei casi d'uso (in inglese \emph{Use Case Diagram}) sono diagrammi di tipo \gls{uml} dedicati alla descrizione delle funzioni o servizi offerti da un sistema, così come sono percepiti e utilizzati dagli attori che interagiscono col sistema stesso.

\subsection{Nomenclatura del codice identificativo}

%\subparagraph{Denominazione }
Ogni codice presenta una forma del tipo:

\begin{center}
	\texttt{UC[Numero]-[Servizio]}
\end{center}

\begin{itemize}
	\item \textbf{UC}: sta per \textit{use case}, l'equivalente inglese di ``caso d'uso''.
	\item \textbf{Numero}: numero progressivo che si riferisce al caso. Se il caso d'uso è principale, è un semplice intero (e.g. 1 per il primo caso d'uso). Mentre se è un sotto caso, presenta due interi separati da un punto (e.g 1.1 per per il primo figlio del primo caso d'uso);
	\item \textbf{Servizio}: indica il servizio preso come riferimento per il caso d'uso.
	\begin{itemize}
		\item \textbf{ES}: Email Sender.
		\item \textbf{L}: User/Login.
		\item \textbf{C}: Catalog.
		\item \textbf{A}: Applicant.
	\end{itemize}
\end{itemize}

\subsection{Casi d'uso relativi al microservizio Email Sender}

\begin{usecase}{1}{ES}{Aggiunta proprietà per invio mail}
	\usecaseactors{Sviluppatore SyncRec}
	
	\usecasepre{una nuova proprietà va aggiunta al servizio Email Sender}
	
	\usecasedesc{lo sviluppatore vuole aggiungere una nuova proprietà nel servizio}
	
	\begin{ucscenarioprincipale}
		\item L'utente inserisce la chiave da inserire;
		\item L'utente inserisce il valore della proprietà.
	\end{ucscenarioprincipale}

	\usecasepost{una nuova proprietà è stata aggiunta con successo al servizio}
	
	\begin{ucestensioni}
		\item UC2-ES
	\end{ucestensioni}

	\begin{ucgeneralizzazioni}
		\item UC3-ES
		\item UC4-ES
		\item UC5-ES
	\end{ucgeneralizzazioni}

	\label{uc:aggiunta-proprieta-es}
\end{usecase}

\begin{usecase}{2}{ES}{Visualizzazione errore inserimento proprietà}
	\usecaseactors{Sviluppatore SyncRec}
	
	\usecasepre{c'è un tentativo di aggiunta di una proprietà}
	
	\usecasedesc{lo sviluppatore viene avvisato che c'è stato un tentativo di inserimento non valido: chiave non riconosciuta oppure valore non valido}
	
	\begin{ucscenarioprincipale}
		\item Lo sviluppatore visualizza il messaggio d'errore.
	\end{ucscenarioprincipale}
	
	\usecasepost{il messaggio d'errore è stato visualizzato}
	
	\label{uc:vis-errore-ins-proprieta-es}
\end{usecase}

\begin{usecase}{3}{ES}{Aggiunta proprietà ``sender''}
	\usecaseactors{Sviluppatore SyncRec}
	
	\usecasepre{la proprietà ``sender'' va aggiunta al servizio Email Sender}
	
	\usecasedesc{lo sviluppatore vuole aggiungere la proprietà sender nel servizio Email Sender}

	\begin{ucscenarioprincipale}
		\item Lo sviluppatore seleziona la chiave ``sender'';
		\item Lo sviluppatore inserisce il valore della proprietà ``sender''.
	\end{ucscenarioprincipale}

	\usecasepost{la proprietà ``sender'' è stata aggiunta con successo}
	
	\label{uc:aggiunta-sender-es}
\end{usecase}

\begin{usecase}{4}{ES}{Aggiunta proprietà ``password''}
	\usecaseactors{Sviluppatore SyncRec}
	
	\usecasepre{la proprietà ``password'' va aggiunta al servizio Email Sender}
	
	\usecasedesc{lo sviluppatore vuole aggiungere la proprietà ``password'' relativa al ``sender'' nel servizio Email Sender. Il servizio salverà la password criptandola con un opportuno algoritmo}

	\begin{ucscenarioprincipale}
		\item Lo sviluppatore seleziona la chiave ``password'';
		\item Lo sviluppatore inserisce il valore della proprietà ``password''.
	\end{ucscenarioprincipale}

	\usecasepost{la proprietà ``password'' è stata aggiunta con successo}
	\label{uc:aggiunta-password-es}
\end{usecase}

\begin{usecase}{5}{ES}{Aggiunta proprietà ``subject''}
	\usecaseactors{Sviluppatore SyncRec}
	
	\usecasepre{la proprietà ``subject'' va aggiunta al servizio Email Sender}
	
	\usecasedesc{lo sviluppatore vuole aggiungere la proprietà ``subject'', l'oggetto della email da inviare, nel servizio Email Sender}
	
	\begin{ucscenarioprincipale}
		\item Lo sviluppatore seleziona la chiave ``subject'';
		\item Lo sviluppatore inserisce il valore della proprietà ``subject''.
	\end{ucscenarioprincipale}
	
	\usecasepost{la proprietà ``subject'' è stata aggiunta con successo}
	
	\label{uc:aggiunta-subject-es}
\end{usecase}


\begin{usecase}{6}{ES}{Modifica proprietà per invio mail}
	\usecaseactors{Sviluppatore SyncRec}
	
	\usecasepre{una proprietà necessita di essere modificata}
	
	\usecasedesc{lo sviluppatore vuole modificare il valore di una nuova proprietà nel servizio}
	
	\begin{ucscenarioprincipale}
		\item Lo sviluppatore inserisce la chiave da modificare;
		\item Lo sviluppatore inserisce il nuovo valore della proprietà.
	\end{ucscenarioprincipale}
	
	\usecasepost{una proprietà è stata modificata con successo}
	
	\begin{ucestensioni}
		\item UC7-ES
	\end{ucestensioni}
	
	\begin{ucgeneralizzazioni}
		\item UC8-ES
		\item UC9-ES
		\item UC10-ES
	\end{ucgeneralizzazioni}
	
	\label{uc:modifica-proprieta-es}
\end{usecase}

\begin{usecase}{7}{ES}{Visualizzazione errore modifica proprietà}
	\usecaseactors{Sviluppatore SyncRec}

	\usecasepre{c'è un tentativo di modifica di una proprietà}

	\usecasedesc{lo sviluppatore viene avvisato che c'è stato un tentativo di modifica non valido: chiave non riconosciuta oppure valore non valido}

	\begin{ucscenarioprincipale}
		\item Lo sviluppatore visualizza il messaggio d'errore.
	\end{ucscenarioprincipale}

	\usecasepost{il messaggio d'errore è stato visualizzato}
	
	\label{uc:vis-errore--mod-proprieta-es}
\end{usecase}

\begin{usecase}{8}{ES}{Modifica proprietà ``sender''}
	\usecaseactors{Sviluppatore SyncRec}
	
	\usecasepre{la proprietà ``sender'' necessita di essere modificata}
	
	\usecasedesc{lo sviluppatore vuole modificare la proprietà ``sender'' nel servizio Email Sender}
	
	\begin{ucscenarioprincipale}
		\item Lo sviluppatore seleziona la modifica della chiave ``sender'';
		\item Lo sviluppatore inserisce il nuovo valore della proprietà ``sender''.
	\end{ucscenarioprincipale}
	
	\usecasepost{la proprietà ``sender'' è stata modificata con successo}
	
	\label{uc:modifica-sender-es}
\end{usecase}

\begin{usecase}{9}{ES}{Modifica proprietà ``password''}
	\usecaseactors{Sviluppatore SyncRec}
	
	\usecasepre{la proprietà ``password'' necessita di essere modificata}
	
	\usecasedesc{lo sviluppatore vuole modificare la proprietà ``password'' relativa al ``sender'' nel servizio. Il servizio salverà la nuova password criptandola con un opportuno algoritmo}
	
	\begin{ucscenarioprincipale}
		\item Lo sviluppatore seleziona la modifica della chiave ``password'';
		\item Lo sviluppatore inserisce il nuovo valore della proprietà ``password''.
	\end{ucscenarioprincipale}
	
	\usecasepost{la proprietà ``password'' è stata modificata con successo}
	
	\label{uc:modifica-password-es}
\end{usecase}

\begin{usecase}{10}{ES}{Modifica proprietà ``subject''}
	\usecaseactors{Sviluppatore SyncRec}
	
	\usecasepre{la proprietà ``subject'' necessita di essere modificata}
	
	\usecasedesc{lo sviluppatore vuole modificare la proprietà ``subject'', ossia l'oggetto della email da inviare, nel servizio Email Sender}

	\begin{ucscenarioprincipale}
		\item Lo sviluppatore seleziona la modifica della chiave ``subject'';
		\item Lo sviluppatore inserisce il nuovo valore della proprietà ``subject''.
	\end{ucscenarioprincipale}
	
	\usecasepost{la proprietà ``subject'' è stata modificata con successo}
	
	\label{uc:modifica-subject-es}
\end{usecase}

\begin{usecase}{11}{ES}{Invio email}
	\usecaseactors{Front-end}
	
	\usecasepre{Il \textit{front-end} deve inviare una email in seguito a determinate azioni dell'utente che usa la piattaforma}
	
	\usecasedesc{il \textit{front-end}, sotto determinate condizioni (come ad esempio la fine della compilazione dello \gls{skill-matrix}), ha necessità di inviare un email automatica specifica al candidato, contenente link e informazioni utili}
	
	\begin{ucscenarioprincipale}
		\item Il \textit{front-end} invoca le \acrshort{api} per l'invio dell'email.
	\end{ucscenarioprincipale}
	
	\usecasepost{l'email è stata inviata con successo}
	
	\begin{ucestensioni}
		\item UC12-ES
	\end{ucestensioni}

	\label{uc:invio-email-es}
\end{usecase}

\begin{usecase}{12}{ES}{Risposta contenente un errore}
	\usecaseactors{\textit{front-end}}
	
	\usecasepre{Una email tenta di essere inviata tramite il servizio Email Sender}
	
	\usecasedesc{Una risposta che spieghi l'errore viene inviata al front-end come risposta se la chiamata dovesse fallire}

	\begin{ucscenarioprincipale}
		\item Il \textit{front-end} riceve la risposta d'errore.
	\end{ucscenarioprincipale}
	
	\usecasepost{Il messaggio d'errore è stato ricevuto con successo}
	
	\label{uc:vis-errore-invio-es}
\end{usecase}

% ****************************

\subsection{Casi d'uso relativi al microservizio Login}


\begin{usecase}{1}{L}{Aggiunta nuovo utente}
	\usecaseactors{\textit{front-end}}
	
	\usecasepre{Un nuovo utente dev'essere inserito nel servizio di Login}
	
	\usecasedesc{il \textit{front-end} deve inserire un nuovo utente}
	
	\begin{ucscenarioprincipale}
		\item Il \textit{front-end} inserisce i dati relativi al nuovo utente;
		\item Il \textit{front-end} chiama la funzionalità di inserimento nuovo utente del servizio di Login.
	\end{ucscenarioprincipale}

	\usecasepost{Un nuovo utente è stato inserito con successo}

	\begin{ucestensioni}
		\item UC3-L
	\end{ucestensioni}

	\begin{ucgeneralizzazioni}
		\item UC2-L
	\end{ucgeneralizzazioni}

	\label{uc:aggiunta-utente-l}
\end{usecase}

\begin{usecase}{2}{L}{Aggiunta nuovo utente con richiesta HTTP}
	\usecaseactors{\textit{front-end}}

	\usecasepre{Un nuovo utente dev'essere inserito nel servizio di Login}

	\textbf{\\Descrizione:} il \textit{front-end} deve inserire un nuovo utente. Per fare ciò,
	chiama le opportune \acrshort{api} esposte dal servizio di Login. La richiesta dovrà contenere i campi:
	\begin{itemize}[noitemsep]
		\item username;
		\item email;
		\item password.
	\end{itemize}
	E potrà contenere i campi opzionali:
	\begin{itemize}[noitemsep]
		\item role;
		\item name;
		\item surname;
		\item office.
	\end{itemize}

	\begin{ucscenarioprincipale}
		\item Il \textit{front-end} inserisce i dati relativi al nuovo utente;
		\item Il \textit{front-end} chiama la funzionalità di inserimento nuovo utente del servizio di Login.
	\end{ucscenarioprincipale}
	
	\usecasepost{Un nuovo utente è stato inserito con successo}
	
	\begin{ucestensioni}
		\item UC3-L
	\end{ucestensioni}

	\label{uc:richiesta-aggiunta-utente-l}
\end{usecase}

\begin{usecase}{3}{L}{Visualizzazione errore inserimento utente}
	\usecaseactors{\textit{front-end}}
	
	\usecasepre{c'è un tentativo di aggiunta di un nuovo utente}
	
	\usecasedesc{viene inviata una risposta al \textit{front-end} contenente un messaggio d'errore per tentativo di inserimento utente non valido}

	\begin{ucscenarioprincipale}
		\item il \textit{front-end} riceve il messaggio d'errore.
	\end{ucscenarioprincipale}

	\usecasepost{il messaggio d'errore è stato ricevuto dal \textit{front-end}}
	
	\label{uc:vis-errore-ins-utente-l}
\end{usecase}

\begin{usecase}{4}{L}{Modifica utente}
	\usecaseactors{\textit{front-end}}
	
	\usecasepre{Un utente già presente dev'essere modificato}
	
	\usecasedesc{il \textit{front-end} deve modificare un utente}
	
	\begin{ucscenarioprincipale}
		\item Il \textit{front-end} inserisce i dati relativi all'utente da modificare;
		\item Il \textit{front-end} chiama la funzionalità di modifica dell'utente del servizio di Login.
	\end{ucscenarioprincipale}

	\usecasepost{Un utente è stato modificato con successo}

	\begin{ucestensioni}
		\item UC6-L
	\end{ucestensioni}
	
	\begin{ucgeneralizzazioni}
		\item UC5-L
	\end{ucgeneralizzazioni}
	
	\label{uc:modifica-utente-l}
\end{usecase}

\begin{usecase}{5}{L}{Modifica utente con richiesta HTTP}
	\usecaseactors{\textit{front-end}}
	
	\usecasepre{Un utente già presente nel servizio di Login dev'essere modificato}
	
	\textbf{\\Descrizione:} il \textit{front-end} deve modificare un utente. Per fare ciò,
	chiama le opportune \acrshort{api} esposte dal servizio di Login. La richiesta potrà contenere una qualsiasi combinazione dei campi:
	\begin{itemize}[noitemsep]
		\item username;
		\item email;
		\item password;
		\item role;
		\item name;
		\item surname;
		\item office.
	\end{itemize}

	\begin{ucscenarioprincipale}
		\item Il \textit{front-end} inserisce i dati relativi all'utente che vuole modificare;
		\item Il \textit{front-end} chiama le \acrshort{api} di inserimento nuovo utente del servizio di Login.
	\end{ucscenarioprincipale}

	\usecasepost{Un utente è stato modificato con successo}

	\begin{ucestensioni}
		\item UC6-L
	\end{ucestensioni}
	
	\label{uc:richiesta-modifica-utente-l}
\end{usecase}

\begin{usecase}{6}{L}{Visualizzazione errore modifica utente}
	\usecaseactors{\textit{front-end}}

	\usecasepre{c'è un tentativo di modifica di un utente}
	
	\usecasedesc{viene inviata una risposta al \textit{front-end} contenente un messaggio d'errore per tentativo di validazione dei dati relativi all'utente da modificare}

	\begin{ucscenarioprincipale}
		\item il \textit{front-end} riceve il messaggio d'errore.
	\end{ucscenarioprincipale}

	\usecasepost{il messaggio d'errore è stato ricevuto dal \textit{front-end}}

	\label{uc:vis-errore-modifica-utente-l}
\end{usecase}


\begin{usecase}{7}{L}{Rimozione utente}
	\usecaseactors{\textit{front-end}}

	\usecasepre{Un utente già presente dev'essere rimosso dal servizio}

	\usecasedesc{il \textit{front-end} deve rimuovere un utente}

	\begin{ucscenarioprincipale}
		\item Il \textit{front-end} inserisce l'ID relativo all'utente da modificare;
		\item Il \textit{front-end} chiama la funzionalità di rimozione utente del servizio di Login.
	\end{ucscenarioprincipale}

	\usecasepost{Un utente è stato rimosso con successo}

	\begin{ucestensioni}
		\item UC9-L
	\end{ucestensioni}

	\begin{ucgeneralizzazioni}
		\item UC8-L
	\end{ucgeneralizzazioni}

	\label{uc:rimozione-utente-l}
\end{usecase}

\begin{usecase}{8}{L}{Rimozione utente con richiesta HTTP}
	\usecaseactors{\textit{front-end}}

	\usecasepre{Un utente già presente nel servizio di Login dev'essere rimosso dal servizio}

	\textbf{\\Descrizione:} il \textit{front-end} deve rimuovere un utente. Per fare ciò,
	chiama le opportune \acrshort{api} esposte dal servizio di Login. La richiesta conterrà l'ID dell'utente da eliminare.

	\begin{ucscenarioprincipale}
		\item Il \textit{front-end} inserisce l'ID all'utente che vuole modificare;
		\item Il \textit{front-end} chiama le \acrshort{api} di rimozione utente del servizio di Login.
	\end{ucscenarioprincipale}

	\usecasepost{Un utente è stato rimosso dal servizio con successo}

	\begin{ucestensioni}
		\item UC9-L
	\end{ucestensioni}

	\label{uc:richiesta-rimozione-utente-l}
\end{usecase}

\begin{usecase}{9}{L}{Visualizzazione errore rimozione utente}
	\usecaseactors{\textit{front-end}}

	\usecasepre{c'è un tentativo di rimozione utente}

	\usecasedesc{viene inviata una risposta al \textit{front-end} contenente un messaggio d'errore per ID inserito inesistente}

	\begin{ucscenarioprincipale}
		\item il \textit{front-end} riceve il messaggio d'errore.
	\end{ucscenarioprincipale}

	\usecasepost{il messaggio d'errore è stato ricevuto dal \textit{front-end}}

	\label{uc:vis-errore-rimozione-utente-l}
\end{usecase}

\begin{usecase}{10}{L}{Login}
	\usecaseactors{\textit{front-end}}

	\usecasepre{il \textit{front-end} deve effettuare l'operazione di autenticazione per un utente non riconosciuto}

	\textbf{\\Descrizione:} il \textit{front-end} effettua l'operazione di login tramite le \acrshort{api} esposte dal servizio.
	I campi richiesti per l'operazione sono i seguenti:
	\begin{itemize}[noitemsep]
		\item \texttt{username}: può contenere sia il campo ``username'' dell'utente da autenticare sia il campo ``email'';
		\item \texttt{password}.
	\end{itemize}

	\begin{ucscenarioprincipale}
		\item il \textit{front-end} prepara la richiesta con i campi necessari;
		\item il \textit{front-end} effettua l'operazione di login tramite le \acrshort{api} del servizio.
	\end{ucscenarioprincipale}
	
	\usecasepost{l'operazione di login è avvenuta con successo. Vengono restituiti nella risposta i campi dell'utente autenticato}

	\begin{ucestensioni}
		\item UC11-L
	\end{ucestensioni}

	\label{uc:login-l}
\end{usecase}

\begin{usecase}{11}{L}{Restituzione errore di Login}
	\usecaseactors{\textit{front-end}}
	
	\usecasepre{viene effettuato un tentativo di login}
	
	\usecasedesc{viene inviata una risposta al \textit{front-end} contenente un messaggio d'errore relativo al login}
	
	\begin{ucscenarioprincipale}
		\item il \textit{front-end} riceve il messaggio d'errore.
	\end{ucscenarioprincipale}
	
	\usecasepost{il messaggio d'errore è stato ricevuto dal \textit{front-end}}
	
	\begin{ucgeneralizzazioni}
		\item UC12-L
		\item UC13-L
	\end{ucgeneralizzazioni}

	\label{uc:vis-errore-login-l}

\end{usecase}

\begin{usecase}{12}{L}{Restituzione errore di bad request}
	\usecaseactors{\textit{front-end}}

	\usecasepre{viene effettuato un tentativo di login}

	\usecasedesc{viene inviata una risposta al \textit{front-end} contenente un messaggio d'errore relativo al pessimo tipo di richiesta (\textit{bad request}). Questo può voler dire un tipo di richiesta \acrshort{http} non valida oppure con campi non riconosciuti (e.g. non viene passato il campo ``password'')}

	\begin{ucscenarioprincipale}
		\item il \textit{front-end} riceve il messaggio d'errore riguardante la pessima richiesta.
	\end{ucscenarioprincipale}
	
	\usecasepost{il messaggio d'errore è stato ricevuto dal \textit{front-end}}

	\label{uc:vis-errore-badrequest-l}

\end{usecase}

\begin{usecase}{13}{L}{Restituzione errore di autenticazione fallita}
	\usecaseactors{\textit{front-end}}

	\usecasepre{viene effettuato un tentativo di login}

	\usecasedesc{viene inviata una risposta al \textit{front-end} contenente un messaggio d'errore relativo all'autenticazione fallita. I campi ``email'' e/o ``password'' non corrispondono a nessun utente registrato}

	\begin{ucscenarioprincipale}
		\item il \textit{front-end} riceve il messaggio d'errore riguardante l'autenticazione fallita.
	\end{ucscenarioprincipale}

	\usecasepost{il messaggio d'errore è stato ricevuto dal \textit{front-end}}

	\label{uc:vis-errore-auth-l}

\end{usecase}

% ****************************

\subsection{Casi d'uso relativi al microservizio Catalog}



% ****************************
% ****************************
% ****************************

\section{Tracciamento dei requisiti}

Da un'attenta analisi dei requisiti e degli use case effettuata sul progetto è stata stilata la tabella che traccia i requisiti in rapporto agli use case.\\
Sono stati individuati diversi tipi di requisiti e si è quindi fatto utilizzo di un codice identificativo per distinguerli.\\
Il codice dei requisiti è così strutturato R(F/Q/V)(N/D/O) dove:
\begin{enumerate}
	\item[R =] requisito
    \item[F =] funzionale
    \item[Q =] qualitativo
    \item[V =] di vincolo
    \item[N =] obbligatorio (necessario)
    \item[D =] desiderabile
    \item[Z =] opzionale
\end{enumerate}
Nelle tabelle \ref{tab:requisiti-funzionali}, \ref{tab:requisiti-qualitativi} e \ref{tab:requisiti-vincolo} sono riassunti i requisiti e il loro tracciamento con gli use case delineati in fase di analisi.

\newpage

\begin{table}%
\caption{Tabella del tracciamento dei requisti funzionali}
\label{tab:requisiti-funzionali}
\begin{tabularx}{\textwidth}{lXl}
\hline\hline
\textbf{Requisito} & \textbf{Descrizione} & \textbf{Use Case}\\
\hline
RFN-1     & L'interfaccia permette di configurare il tipo di sonde del test & UC1 \\
\hline
\end{tabularx}
\end{table}%

\begin{table}%
\caption{Tabella del tracciamento dei requisiti qualitativi}
\label{tab:requisiti-qualitativi}
\begin{tabularx}{\textwidth}{lXl}
\hline\hline
\textbf{Requisito} & \textbf{Descrizione} & \textbf{Use Case}\\
\hline
RQD-1    & Le prestazioni del simulatore hardware deve garantire la giusta esecuzione dei test e non la generazione di falsi negativi & - \\
\hline
\end{tabularx}
\end{table}%

\begin{table}%
\caption{Tabella del tracciamento dei requisiti di vincolo}
\label{tab:requisiti-vincolo}
\begin{tabularx}{\textwidth}{lXl}
\hline\hline
\textbf{Requisito} & \textbf{Descrizione} & \textbf{Use Case}\\
\hline
RVO-1    & La libreria per l'esecuzione dei test automatici deve essere riutilizzabile & - \\
\hline
\end{tabularx}
\end{table}%