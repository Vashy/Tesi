% !TEX encoding = UTF-8
% !TEX TS-program = pdflatex
% !TEX root = ../tesi.tex

%**************************************************************
\chapter{Conclusioni}\label{cap:conclusioni}

%**************************************************************

%**************************************************************
\section{Consuntivo finale}

Lo stage previsto dal percorso di laurea a Padova in informatica valgono per un totale di 11 \gls{cfu}, corrispondenti a 300-320 ore di lavoro. Tra queste, sono incluse le ore di formazione della prima parte dello stage.

Le attività si sono svolte dal 6 maggio al 28 giugno, per un totale di 320 ore. Tra queste, sono stato assente in due giornate, portando il totale delle ore a 304.

%**************************************************************
\section{Raggiungimento degli obiettivi}
Gli obiettivi obbligatori, desiderabili e facoltativi riportati in \S\ref{obiettivi-stage} sono stati interamente soddisfatti. L'unico obiettivo facoltativo richiede lo studio del \textit{framework} \gls{spring-cloud}, di cui viene data un'infarinatura in \S\ref{cap:spring-cloud}.

Per quanto riguarda i requisiti riportati in \S\ref{tracciamento-requisiti}, sono stati soddisfatti tutti i requisiti funzionali obbligatori e desiderabili e buona parte di quelli desiderabili. 
I requisiti facoltativi non soddisfatti sono riportati nella tabella che segue:

\begin{table}[H]
	\centering
	\begin{paddedtablex}[1.4]{0.5\textwidth}{XX}
		\textbf{Requisito} & \textbf{Caso d'uso}\\
		\toprule
		RFF-47 & \ref{uc:vis-applicant-phone-http-a}\\
		RFF-48 & \ref{uc:vis-errore-applicant-a}\\
		\bottomrule
%		\reqrow{RFF}{Il servizio Applicant permette di visualizzare un applicant con un numero di telefono specificato tramite richiesta HTTP}{\ref{uc:vis-applicant-phone-http-a}}
%		\reqrow{RFF}{Il servizio Applicant mostra un opportuno errore nel caso un applicant ricercato non sia presente}{\ref{uc:vis-errore-applicant-a}}
	\end{paddedtablex}
	\vspace{4pt}
	\caption{Elenco dei requisiti non soddisfatti}
\end{table}

%**************************************************************
\section{Conoscenze acquisite}

Il percorso di tirocinio mi ha permesso di acquisire buone conoscenze su \gls{java} e sul \textit{framework} \gls{spring}, un'accoppiata che svolge un ruolo centrale nel panorama aziendale \gls{it} italiano e non.
C'è stata anche una parte di formazione su alcune tecnologie, quali \gls{angular}, e di studio, in particolare sui \glsdisp{microservizio}{microservizi} e su \gls{spring-cloud}, entrambi moderni.

%**************************************************************
\section{Valutazione personale}

Le conoscenze pregresse fornitemi dal percorso di laurea sono state essenziali per poter svolgere un tirocinio orientato su tecnologie e paradigmi così attuali.

In particolare, mi sono state utili le conoscenze di programmazione acquisite durante i tre corsi relativi (\textit{Programmazione}, \textit{Programmazione ad oggetti}, \textit{Programmazione concorrente e distribuita}) e \textit{Ingegneria del software} con relativo progetto, in cui ho messo in pratica diversi concetti utili per questo tirocinio, tra cui i microservizi, la \gls{dependency-injection},  i \gls{design-pattern} e acquisito anche un approccio rigoroso alla metodologia di lavoro.

Il percorso di stage mi ha permesso di fissare e approfondire tali competenze, rendendole valide per il personale profilo
lavorativo.
